\documentclass[11pt, noincludeaddress, nopagination]{classes/cthit}
\usepackage{titlesec}
\usepackage{verbatimbox}
\usepackage{tabularx}
\usepackage{changepage}

\titleformat{\paragraph}[hang]{\normalfont\normalsize\bfseries}{\theparagraph}{1em}{}
\titlespacing*{\paragraph}{0pt}{3.25ex plus 1ex minus 0.2ex}{0.7em}

\graphicspath{ {images/} }

\begin{document}

\title{Möteshållning för sektionsmöte över Zoom}
\authors{styrIT}

\titlelabel{\S \thetitle\quad}

\revisioned{2021--02--41}
\maketitle

\thispagestyle{empty}

\newpage

\makeheadfoot%

%Sidnumreringsstart
\setcounter{page}{1}

\vspace{-20pt}
\subsection*{Bakgrund}
Då Zoom som plattform inte ger oss samma förutsättningar som en hörsal behöver vi definiera en möteshållning för att vi ska kunna ta oss igenom sektionsmötet så smidigt som möjligt.

Härefter följer regler, rekommendationer och upplysningar som tillsammans uppgör möteshållningen. Dessa är framtagna för att förenkla sektionsmötet. Dokumentet skall användas tillsammans med \href{https://styrit.chalmers.it/wp-content/uploads/sektionsmoteshandbok.pdf}{sektionsmöteshandboken} där reglerna i detta dokument kommer ha företräde framför handboken.

\subsection*{Allmänt}
\begin{itemize}

    \item Varje deltagare får endast vara ansluten på sektionsmötet med en enhet.
    \item Alla deltagare behöver vara anslutna till sin egen enhet och inte dela med någon annan.
    \item Då Zooms ''Poll'' funktion inte fungerar i webbläsaren rekommenderas närvarande på mötet att använda en Zoomklient (mobil- eller desktopapplikationen).
    \item Användarnamnet för varje deltagare måste innehålla deltagarens för- och efternamn.
    \item För allas trevnad uppmanas alla som har tillgång till kamera att använda denna.
\end{itemize}

\subsection*{Begäran av ordet}
\begin{itemize}
    \item Begäran av ordet sker via \href{https://talarlista.chalmers.it/user}{talarlista.chalmers.it}
    \item Begäran av ordet vid Sakupplysningar och Ordningsfrågor kan även göras genom att skriva direkt på Zoom, alternativ till någon i styrIT på Slack.
    \item Vid ändringsyrkanden kan man skriva direkt till någon i styrIT på Zoom eller Slack med yrkandet.
    \item Replik kan tilldelas direkt av mötets ordförande, vid behov kan det begäras genom att skriva direkt på Zoom, exempelvis \textit{Replik som ArmIT}.
\end{itemize}

\subsection*{Röstning}

\begin{adjustwidth}{0.3 cm}{0cm}
\subsubsection*{Acklamation \& votering}
Acklamation kommer vid enklare situationer ske direkt genom ''Yes'' (Grön checkbox) och ''No'' (Rött kryss) i Zooms Participant-lista, och i andra fall genom polls. Vid önskemål om votering, ger båda dessa alternativ möjlighet för rösträkning.

Talman kommer specificera vilket som gäller för varje specifikt fall.

\subsubsection*{Sluten votering}
Om sluten votering begärs kommer röstsystemet voteIT (ITs version) att användas. Personliga koder kommer i förstahand mejlas av mötets justerare till de som är närvarande i Zoommötet. Sluten votering kan begäras genom talarlistan.
\end{adjustwidth}

\subsection*{Personinval}
Vid inval kommer de nominerade att placeras i ett breakout room tillsammans med en representant från styrIT (istället för att lämna rummet som vid ett fysiskt sektionsmöte). Vänligen respektera att du som nominerad inte ska vara vid hörselavstånd (fysiskt \& remote) till övriga mötesdeltagare vid dessa tillfällen, samt att du stannar kvar i breakout room tills du får förfrågan om att återgå till huvudmötesrummet.


\end{document}

