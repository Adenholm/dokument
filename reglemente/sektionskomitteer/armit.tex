\subsection{\ARMITFULL{}}
\subsubsection{Sammansättning}
\ARMIT{} består av ordförande, kassör, sekreterare samt 1-5 övriga ledamöter.

\subsubsection{Inval}
\ARMIT{}s medlemmar skall varje år väljas på det första ordinarie vårmötet.

\subsubsection{Det åligger \ARMIT{}}
\begin{att}
	\item Förmedla information rörande arbetsmarknadsfrågor till studenter.
	\item Förmedla en bred och representativ bild av arbetsmarknaden till studenter.
	\item Främja relationer mellan näringslivet och sektionen samt dess medlemmar.
\end{att}

\subsubsection{Det åligger \ARMIT{}s ordförande}
\begin{att}
	\item Leda \ARMIT{}s verksamhet.
\end{att}

\subsubsection{Det åligger \ARMIT{}s kassör}
\begin{att}
	\item Handha \ARMIT{}s ekonomi.
\end{att}

\subsubsection{Det åligger \ARMIT{}s sekreterare}
\begin{att}
	\item Föra protokoll vid \ARMIT{}s möten.
	\item Handha \ARMIT{}s handlingar.
\end{att}

\subsubsection{Mandatperiod}
Invalda till \ARMIT{} betraktas som en del av \ARMIT{} och sektionen från tillträdet, en (1) dag efter inval, till avträdet, 1:a maj ett år efter tillträdet.

Det yttersta ansvaret för \ARMIT{}s verksamhet och ekonomi övergår till de nyinvalda den 1:a maj.
