\section{Sektionskommittéer}

\subsection{Förteckning}
Teknologsektionen har de sektionskommittéer som finns i tabell \ref{kommittetabell}.

\begin{table}[h!]
\caption{\label{kommittetabell} Förteckning över sektionskommittéer.}
\centering
\renewcommand{\arraystretch}{1.4}
\begin{tabular}{l|c|c|c}
\textbf{Kommitté} & \multicolumn{1}{l|}{\textbf{\begin{tabular}[c]{@{}l@{}}\phantom{l}Ledamöter\phantom{l}\end{tabular}}} & \multicolumn{1}{l|}{\textbf{\phantom{l}Inval \phantom{l}}} & \multicolumn{1}{l}{\textbf{\phantom{l}Mandat\phantom{l}}}   \\ \hline
\ARMIT       & \phantom{*} 1 - 5 *    & LP3     & 1/5       	\\ \hline
\DIGIT       & 1 - 6                  & LP3     & 1/5       	\\ \hline
\EQUALIT     & 0 - 3                  & LP4     & 1/7			\\ \hline
\FANBARERIT  & 0 - 2                  & LP4     & 1/7       	\\ \hline
\FLASHIT     & 0 - 4                  & LP3     & 1/5			\\ \hline
\FRITID      & 1 - 4                  & LP4     & 1/7       	\\ \hline
\MRCIT       & 2 - 5                  & LP2     & 1/1       	\\ \hline
\NOLLKIT     & 2 - 6                  & LP2     & 1/1       	\\ \hline
\PRIT        & 2 - 6                  & LP2     & 1/1       	\\ \hline
\SEXIT       & 3 - 6                  & LP2     & 1/1           \\ \hline
\TRADGARDSMASTERIT & 0-2              & LP1     & 1/11          
\end{tabular}
\renewcommand{\arraystretch}{1}
\end{table}    

I \ref{ledmot} till \ref{mandat} finns beskrivning om vad varje kolonn betyder. 

\subsubsection{Ledamöter} \label{ledmot}
Antal ledamöter som kan ingå i en kommitté finns i tabell \ref{kommittetabell} för respektive kommitté. Kommittén består av dessa ledamöter samt ordförande och kassör. 

\ARMIT{} har dessutom en sekreterare som är ålagd att handha \ARMIT{}s handlingar. 

\subsubsection{Inval}
En kommitté väljs in under ordinarie sektionsmöte den läsperiod som står i tabell \ref{kommittetabell}. 

\subsubsection{Mandat} \label{mandat}
Det yttersta ansvaret för en kommittés verksamhet och ekonomi övergår till de nyinvalda vid det datum som står för kommittén i tabell \ref{kommittetabell}.

\subsection{Påstigning}

Invalda till en kommitté betraktas som en del av kommittén och sektionen från två läsdagar efter invalet, tills då de inte längre har mandat.

\subsection{Åligganden för alla kommittéer}
Det åligger samtliga sektionskommittéer att framlägga förslag till valberedningen på efterträdare till respektive sektionskommittéer. 

\subsubsection{Det åligger varje kommittés ordförande}
\begin{att}
	\item Leda kommitténs verksamhet.
	\item Handha kommitténs handlingar. (Gäller ej \ARMIT{}.)
\end{att}

\subsubsection{Det åligger varje kommittés kassör}
\begin{att}
	\item Handha kommitténs ekonomi.
\end{att}

\subsection{Kommittéspecifika åligganden}
Förutom de åligganden som gäller alla kommittéer finns åligganden som är specifika för varje kommitté.

\subsubsection{Det åligger \ARMIT{}}
\begin{att}
	\item Förmedla information rörande arbetsmarknadsfrågor till studenter.
	\item Förmedla en bred och representativ bild av arbetsmarknaden till studenter.
	\item Främja relationer mellan näringslivet och sektionen samt dess medlemmar.
\end{att}

\subsubsection{Det åligger \DIGIT}
\begin{att}
	\item Underhålla och utveckla system för sektionen som tillsammans uppfyller följande behov:
	\begin{itemize}
		\item En webbplats som kan presentera nyheter och sponsorer/partner.
		\item Ett system som sektionsmötet kan använda för att genomföra sluten votering
		\item Ett system som ser till att alla organ inom sektionen får tillgång till en epostadress som kan nyttjas av organet
		\item Ett system som låter gemene IT-teknolog eller organ på sektionen boka delar av, eller hela sektionslokalen
		\item Regelbundna säkerhetskopior på data från de system och tjänster som tillgodoser kraven i denna listan
	\end{itemize}
	\item Tillsammans med sektionens rustmästeri underhålla och utveckla de digitala system som finns i sektionslokalen.
	\item Tillhandaha ett medlemsregister med information om vilka som för tillfället är invalda i sektionens kommittéer, föreningar och andra instanser.
	\item Uppmuntra till en god kodkultur på sektionen
\end{att}

\subsubsection{Det åligger \EQUALIT{}}
\begin{att}
    \item arrangera minst 1 arrangemang med jämlikhetsfokus per läsår.
    \item vara SAMO behjälplig i jämlikhetsarbete.
    \item bistå sektionen med utbildning i jämlikhet.
\end{att}

\subsubsection{Det åligger \FANBARERIT}
\label{sec:fanbarerit:function}
\begin{att}
	\item Tillse att sektionens fana är representerad på
	\begin{itemize}
		\item samtliga ordinarie sektionsmöten
		\item vårens och höstens mösspåtagningar
		\item Chalmerscortègen
		\item första mottagningsdagen
		\item av sektionen anordnade finare tillställningar
		\item övriga av Chalmers studentkårs marskalksämbetes kallade tillställningar.
	\end{itemize}
	\item Vid de ovanstående tillställningar, som sektionens fana är representerad, iträda
	\begin{itemize}
		\item högtidsdräkt.
		\item vit Chalmersmössa.
		\item minst Chalmers studentkårs samt sektionens frackband.
	\end{itemize}
	\item En gång per termin arrangera ett finare arrangemang av mindre storlek som främjar kultur.
	\item Tillhandaha sektionens frackband och pin.
\end{att}

\subsubsection{Det åligger \FLASHIT}
\begin{att}
	\item vid minst ett tillfälle per verksamhetsår förse sektionens styrelser, arbetsgrupper, nämnder, kommittéer och intresseföreningar med profileringsbilder om så efterfrågas av dessa.
	\item förse sektionens styrelser, nämnder och kommittéer med fotografier från minst ett av deras arrangemang per verksamhetsår om så efterfrågas av dessa. Vid sådan fotografering ansvarar arrangör för att ge fotograf tillgång till arrangemanget.
	\item uppmuntra och bidra till ökat fotograferingsintresse på sektionen.
\end{att}

\subsubsection{Det åligger \FRITID}
\begin{att}
	\item Ta ut lag till Chalmersmästerskap i olika idrotter.
	\item Varje läsperiod arrangera ett evenemang av idrottslig anda.
	\item Främja fysiska aktivititer som speglar sektionens medlemmars intresse.
\end{att}

\subsubsection{Det åligger \MRCIT}
\begin{att}
  \item Planera och leda mottagningsaktiviteter för studenterna som börjar på masterprogrammen Interaction Design and Technologies och Software Engineering and Technology.
\end{att}

\subsubsection{Det åligger \NOLLKIT}
\begin{att}
	\item Planera och leda sektionens mottagning för de nyantagna IT-teknologerna.
\end{att}

\subsubsection{Det åligger \PRIT}
\begin{att}
	\item Ansvara för att det anordnas arrangemang för medlemmar som främjar sammanhållningen på sektionen.
	\item Arrangera en pub under pubrundan varje läsperiod.
	\item Driva representationen av IT:s skyddshelgon i form av att utse representant för maskot till större centrala arrangemang där det anses passande.
	\item Underhålla IT:s märke ovanför Olgas trappor.
	\item Tillhandahålla sektionens tygmärke.
	\item Organisera arbetet med städning, underhåll och nybyggnation av sektionslokalen.
	\item Ansvara för sektionslokalens inventarier.
\end{att}

\subsubsection{Det åligger \SEXIT}
\begin{att}
	\item En gång om året anordna en finsittning för utdelning av pedagogiska priset.
	\item En gång per läsperiod arrangera i Gasquen.
	\item Underhålla sektionens sångbok. 
\end{att}

\subsubsection{Det åligger TrädgårdsmästerIT}
\begin{att}
	\item Vattna, kultivera och sköta om Hubbens växter.
	\item Arrangera minst 1 arrangemang per läsperiod med syfte att öka sektionens engagemang för ekologiska hållbarhetsfrågor eller återkoppla till naturen.
	\item Vara styrIT behjälplig i frågor relaterade till ekologisk hållbarhet.
\end{att}


\begin{comment}

\newpage
\subsection{\SEXITFULL}
\subsubsection{Sammansättning}
\SEXIT{} består av ordförande, kassör samt 3-6 övriga ledamöter.

\subsubsection{Inval}
\SEXIT{}s medlemmar skall varje år väljas på det andra ordinarie höstmötet.

\subsubsection{Det åligger \SEXIT}
\begin{att}
	\item En gång om året anordna en finsittning för utdelning av pedagogiska priset.
	\item En gång per läsperiod arrangera i Gasquen. 
\end{att}

\subsubsection{Det åligger \SEXIT{}s ordförande}
\begin{att}
	\item Leda \SEXIT{}s verksamhet.
	\item Handha \SEXIT{}s handlingar.
	\item Teckna \SEXIT{}s firma.
\end{att}

\subsubsection{Det åligger \SEXIT{}s kassör}
\begin{att}
	\item Handha \SEXIT{}s ekonomi.
	\item Teckna \SEXIT{}s firma.
\end{att}
\newpage
\subsection{\NOLLKITFULL}
\subsubsection{Sammansättning}
\NOLLKIT{} består av ordförande, kassör samt 2-5 övriga ledamöter.

\subsubsection{Inval}
\NOLLKIT{}s medlemmar skall varje år väljas på det andra ordinarie höstmötet.

\subsubsection{Det åligger \NOLLKIT}
\begin{att}
	\item Se till att IT-nollan skolas in till en sann chalmerist.
\end{att}

\subsubsection{Det åligger \NOLLKIT{}s ordförande}
\begin{att}
	\item Leda \NOLLKIT{}s verksamhet.
	\item Handha \NOLLKIT{}s handlingar.
	\item Teckna \NOLLKIT{}s firma.
\end{att}

\subsubsection{Det åligger \NOLLKIT{}s kassör}
\begin{att}
	\item Handha \NOLLKIT{}s ekonomi.
	\item Teckna \NOLLKIT{}s firma.
\end{att}
\newpage
\subsection{\PRITFULL}
\subsubsection{Sammansättning}
\PRIT{} består av ordförande, kassör samt 2-5 övriga ledamöter.

\subsubsection{Inval}
\PRIT{}s medlemmar skall varje år väljas på det andra ordinarie höstmötet.

\subsubsection{Det åligger \PRIT}
\begin{att}
	\item Ansvara för att det anordnas arrangemang för medlemmar som främjar sammanhållningen på sektionen.
	\item Arrangera en pub under pubrundan varje läsperiod.
	\item Se till att studie- och arbetsmiljön i sektionslokalen är god.
	\item Driva representationen av IT:s skyddshelgon i form av att utse representant för maskot till större centrala arrangemang där det anses passande.
	\item Underhålla IT:s märke ovanför Olgas trappor.
	\item Vara mottagningskommittén behjälplig under mottagningen.
	\item Tillhandahålla sektionens tygmärke \& sångbok
	\item Organisera arbetet med städning, underhåll och nybyggnation av sektionslokalen.
	\item Ansvara för sektionslokalens inventarier.
\end{att}

\subsubsection{Det åligger \PRIT{}s ordförande}
\begin{att}
	\item Leda \PRIT{}s verksamhet.
	\item Handha \PRIT{}s handlingar.
	\item Teckna \PRIT{}s firma.
\end{att}

\subsubsection{Det åligger \PRIT{}s kassör}
\begin{att}
	\item Handha \PRIT{}s ekonomi.
	\item Teckna \PRIT{}s firma.
\end{att}


\newpage
\subsection{\FRITIDFULL}
\subsubsection{Sammansättning}
\FRITID{} består av ordförande, kassör samt 1-4 övriga ledamöter.

\subsubsection{Inval}
\FRITID{}s medlemmar skall varje år väljas på det andra ordinarie vårmötet.

\subsubsection{Det åligger \FRITID}
\begin{att}
	\item Fungera som en idrottsförening som tillfredställer teknologens idrottsliga behov.
\end{att}

\subsubsection{Det åligger \FRITID{}s ordförande}
\begin{att}
	\item Leda \FRITID{}s verksamhet.
	\item Handha \FRITID{}s handlingar.
	\item Teckna \FRITID{}s firma.
\end{att}

\subsubsection{Det åligger \FRITID{}s kassör}
\begin{att}
	\item Handha \FRITID{}s ekonomi.
	\item Teckna \FRITID{}s firma.
\end{att}
\newpage
\subsection{\ARMITFULL{}}
\subsubsection{Sammansättning}
\ARMIT{} består av ordförande, kassör, sekreterare samt 1-5 övriga ledamöter.

\subsubsection{Inval}
\ARMIT{}s medlemmar skall varje år väljas på det första ordinarie vårmötet.

\subsubsection{Det åligger \ARMIT{}}
\begin{att}
	\item förmedla information rörande arbetsmarknadsfrågor till studenter.
	\item förmedla en bred och representativ bild av arbetsmarknaden till studenter.
	\item främja relationer mellan näringslivet och sektionen samt dess medlemmar.
\end{att}

\subsubsection{Det åligger \ARMIT{}s ordförande}
\begin{att}
	\item leda \ARMIT{}s verksamhet.
\end{att}

\subsubsection{Det åligger \ARMIT{}s kassör}
\begin{att}
	\item handha \ARMIT{}s ekonomi.
\end{att}

\subsubsection{Det åligger \ARMIT{}s sekreterare}
\begin{att}
	\item föra protokoll vid \ARMIT{}s möten.
	\item handha \ARMIT{}s handlingar.
\end{att}

\subsubsection{Mandatperiod}
Invalda till \ARMIT{} betraktas som en del av \ARMIT{} och sektionen från tillträdet, en (1) dag efter inval, till avträdet, 1:a maj ett år efter tillträdet.

Det yttersta ansvaret för \ARMIT{}s verksamhet och ekonomi övergår till de nyinvalda den 1:a maj.

\newpage
\subsection{\DIGITFULL}
\subsubsection{Sammansättning}
\DIGIT{} består av ordförande, kassör samt 3-6 ledamöter.

\subsubsection{Inval}
\DIGIT{}s medlemmar skall varje år väljas på det första ordinarie vårmötet.

\subsubsection{Det åligger \DIGIT}
\begin{att}
	\item Underhållande och utveckla sektionens digitala system.
	\item Underhålla och utveckla sektionens digitala kommunikationsplatformar.
	\item Tillsammans med PRIT underhålla och utveckla de digitala system som finns i sektionslokalen.
\end{att}

\subsubsection{Det åligger \DIGIT{}s ordförande}
\begin{att}
	\item Leda \DIGIT{}s verksamhet.
	\item Handha \DIGIT{}s handlingar.
	\item Teckna \DIGIT{}s firma.
\end{att}

\subsubsection{Det åligger \DIGIT{}s kassör}
\begin{att}
	\item Handha \DIGIT{}s ekonomi.
	\item Teckna \DIGIT{}s firma.
\end{att}
\newpage
\subsection{\FANBARERITFULL}
\subsubsection{Sammansättning}
\FANBARERIT{} består av ordförande, kassör samt 0-1 övriga ledamöter.

\subsubsection{Inval}
\FANBARERIT{}s medlemmar skall varje år väljas på det andra ordinarie vårmötet.

\subsubsection{Det åligger \FANBARERIT}
\label{sec:fanbarerit:function}
\begin{att}
	\item tillse att sektionens fana är representerad på
	\begin{itemize}
		\item samtliga ordinarie sektionsmöten.
		\item vårens och höstens mösspåtagningar.
		\item Chalmerscortègen.
		\item första mottagningsdagen.
		\item av sektionen anordnade finare tillställningar.
		\item övriga av Chalmers Studentkårs Marskalksämbets kallade tillställningar.
	\end{itemize}
	\item vid de ovanstående tillställningar, som sektionens fana är representerad, iträda
	\begin{itemize}
		\item högtidsdräkt.
		\item vit Chalmersmössa.
		\item minst Chalmers Studentkårs samt sektionens frackband.
	\end{itemize}
	\item En gång per termin arrangera ett finare arrangemang av mindre storlek som främjar kultur.
\end{att}

\subsubsection{Det åligger \FANBARERIT{}s ordförande}
\begin{att}
	\item Leda \FANBARERIT{}s verksamhet.
	\item Handha \FANBARERIT{}s handlingar.
	\item Teckna \FANBARERIT{}s firma.
\end{att}

\subsubsection{Det åligger \FANBARERIT{}s kassör}
\begin{att}
	\item Handha \FANBARERIT{}s ekonomi.
	\item Teckna \FANBARERIT{}s firma.
\end{att}


\newpage
\subsection{\MRCITFULL}
\subsubsection{Sammansättning}
\MRCIT{} består av ordförande, kassör samt 2-5 övriga ledamöter.

\subsubsection{Inval}
\MRCIT{}s medlemmar skall varje år väljas på det första ordinarie vårmötet.

\subsubsection{Det åligger \MRCIT}
\begin{att}
  \item planera och leda mottagningsaktiviteter för studenterna som börjar på masterprogrammen Interaction Design and Technologies och Software Engineering and Technology.
\end{att}

\subsubsection{Det åligger \MRCIT{}s ordförande}
\begin{att}
  \item leda \MRCIT{}s verksamhet.
  \item handha \MRCIT{}s handlingar.
\end{att}

\subsubsection{Det åligger \MRCIT{}s kassör}
\begin{att}
  \item handha \MRCIT{}s ekonomi.
\end{att}

\subsubsection{Mandatperiod}
Invalda till \MRCIT{} betraktas som en del av \MRCIT{} och sektionen från tillträdet, 1:a mars, till avträdet, 1:a mars ett år efter tillträdet.

Det yttersta ansvaret för \MRCIT{}s verksamhet och ekonomi övergår till de nyinvalda den 1:a mars.
\end{comment}
