\subsection{\FANBARERITFULL}
\subsubsection{Sammansättning}
\FANBARERIT{} består av ordförande, sektionens kassör tillika kassör samt 1-2 övriga ledamöter.

\subsubsection{Inval}
\FANBARERIT{}s medlemmar skall varje år väljas på det andra ordinarie vårmötet.

\subsubsection{Det åligger \FANBARERIT}
\label{sec:fanbarerit:function}
\begin{att}
	\item tillse att sektionens fana är representerad på
	\begin{itemize}
		\item samtliga ordinarie sektionsmöten.
		\item vårens och höstens mösspåtagningar.
		\item Chalmerscortègen.
		\item första mottagningsdagen.
		\item av sektionen anordnade finare tillställningar.
		\item övriga av Chalmers Studentkårs Marskalksämbets kallade tillställningar.
	\end{itemize}
	\item vid de ovanstående tillställningarna iträda
	\begin{itemize}
		\item högtidsdräkt.
		\item vit Chalmersmössa.
		\item minst Chalmers Studentkårs samt sektionens frackband.
	\end{itemize}
\end{att}

\subsubsection{Det åligger \FANBARERIT{}s ordförande}
\begin{att}
	\item Leda \FANBARERIT{}s verksamhet.
	\item Handha \FANBARERIT{}s handlingar.
	\item Teckna \FANBARERIT{}s firma.
\end{att}

%\subsubsection{Samtliga medlemmar i \FANBARERIT{}, utom kassören, har rätt att var och för sig av sektionen få sektionens frackband och pin bekostade.}

%\subsubsection{Samtliga medlemmar i \FANBARERIT{}, utom kassören, har rätt att var för sig efter deltagande i någon av de i \ref{sec:fanbarerit:function} förtecknade tillställningar av sektionen få en kemtvätt av sin högtidsklädsel bekostad vid \FANBARERIT{}s verksamhetsårs slut.}

