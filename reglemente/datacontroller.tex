\section{Datahantering}

\subsection{Definitioner}
	    
	    Med dataägare anses den person vars personuppgifter hanteras av organisationen.
	    
	    Hantering av data inkluderar men är ej begränsat till: lagring, beräkning, använding, visning, distrubution och insamling av data.

\subsection{Personuppgiftsansvarig}

\subsubsection{ Uppdrag}
Personuppgiftsansvarig har ansvar för att datainsamling och datahantering sker i enlighet med allmänna dataskyddsförordningen.  Personuppgiftsansvarig bestämmer hur, och med vilket syfte data får samlas in samt behandlas.

\subsubsection{ Sammansättning}
Består utav 1-5 personer. Om ingen annan väljes blir sektionens ordförande personuppgiftsansvarig.

\subsubsection{ Inval}
Personuppgiftsansvariga skall varje år väljas på det andra ordinarie vårmötet.

\subsubsection{ Rättigheter}
Sektionsstyrelsen äger rätt att i namn och emblem använda sektionens namn och dess symboler.

\subsubsection{Åligganden}
\begin{att}
    \item Utefter bästa förmåga säkerställa att persondata hanteras efter rådande bestämmelser och riktlinje
    \item Ansvara för att datahanterare utbildas.
    \item Ha kunskap om hur data hanteras och lagras.
    \item På uppdrag av en person, lista eller ta bort alla personuppgifter om denne.
\end{att}

\subsubsection{Befogenhet}
\textbf{det bifogas personuppgiftsansvarig}
\begin{att}
    \item utse Datahanterare. Datahanterare uppgift är att hantera och samla in data å personuppgiftsansvarigens vägnar. Det krävs ett skriftligt avtal från personuppgiftsansvarig för att agera som datahanterare.
\end{att}

\subsection{Personuppgiftsombud}

\subsubsection{Uppdrag}
 Personuppgiftsombudet skall fungera som dataägarnas representant inom organisationen och i sin roll kontrollera att rådande bestämmelser samt riktlinjer kring personuppgifter efterföljs.
 
 Personuppgiftsombudet skall även agera som en kontaktperson för dataägarna gentemot sektionen.

\subsubsection{Sammansättning}
1-4 personer kan utses till personuppgiftsombud för sektionen utav sektionsmötet.

Personuppgiftsombud utses för ett år i taget med förvänting om att personuppgioftsombudet skall anmäla intresse och utses två år i följd.

Det är nödvändigt att personuppgiftsombudet kan anses som oberoende och inte har intressen som befinner sig i konflikt till personuppgiftsombudets åligganden.

\subsubsection{Inval}
Personuppgiftsansvariga skall varje år väljas på det första ordinarie höstmötet.
