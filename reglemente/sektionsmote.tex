\section{Sektionsmötet}

\subsection{Kallelse}
Kallelse till sektionsmöte består av mötets
\begin{itemize}
    \item datum
    \item tid
    \item plats.
\end{itemize}

Kallelsen skall publiceras väl synligt för sektionens medlemmar på sektionens anslagstavla, webbsida och sektionslokal. Kallelsen skall anslås korrekt enligt stadgan. 

\subsection{Möteshandlingar}
Förslag till dagordningen skall anslås fem dagar innan sektionsmötet och förslaget skall innehålla:

\begin{itemize}
	\item Datum, plats och tid för mötet
	\item Mötets öppnande
	\item Val av mötets ordförande
	\item Val av mötets sekreterare
	\item Val av mötets justerare tillika rösträknare
	\item Mötets behöriga utlysande
	\item Fastställande av mötets dagordning
	\item Adjungeringar
	\item Föregående mötesprotokoll
	\item Meddelanden
	\item Verksamhetsrapporter
	\item Eventuella års- och revisionsberättelser
	\item Eventuella personval
	\item Propositioner
	\item Motioner
	\item Interpellationer
	\item Övriga frågor
	\item Mötets avslutande
\end{itemize}

\subsection{Mötesordning}
\subsubsection{Begärande av ordet}
Ordet begärs genom handuppräckning och delas i tur och ordning ut av
mötesordförande.
  
\subsubsection{Replik}
Om ett anförande berör en speciell person, har denne rätt till replik om högst
en minut. Replik skall hållas i direkt anslutning till anförandet. En kontrareplik
om högst en minut kan beviljas. Kontra-kontra-replik beviljas ej.
 
\subsubsection{Ordningsfråga}
Debatt i ordningsfråga bryter debatt i sakfråga och skall avgöras innan debatt
i sakfråga fortsätter.

\subsubsection{Streck i debatten}
Streck i debatten behandlas som ordningsfråga. Bifalls yrkandet om streck i debatten, av mötesordföranden, skall mötesordföranden justera talarlistan. Därefter får endast de som står på listan yttra sig i frågan och inga nya yrkanden i sakfrågan får framställas. Upphävande av streck i debatten behandlas även det som ordningsfråga.
 
\subsubsection{Yrkande}
Yrkande framställs muntligt och/eller skriftligt till mötesordförande.

\subsubsection{Reservation}
Reservation mot beslut av sektionsmötet skall anmälas skriftligen senast 24 timmar efter mötet.
 
\subsubsection{Ajournering}
Bifalles yrkandet om ajournering skall tidslängden av ajourneringen fastställas.

\subsubsection{Motion}
Motion som är upptagen på dagordningen måste lyftas och föredragas av motionären eller någon annan på mötet med förslagsrätt, annars faller motionen. Vidare kan sektionsstyrelsen lämna sitt utlåtande om motionen och därefter följer allmän debatt.

\subsubsection{Sluten votering}
När sluten votering begärs skall det digitala röstsystemet VoteIT användas, om inte manuell sluten votering begärs av sektionsmötet via votering, som kan krävas av en sektionsmötesdeltagare. Det är rösträknarnas ansvar att tillhandahålla systemet så att deltagarna kan rösta därigenom. Den senast godkända versionen är:

\href{https://github.com/cthit/VoteIT/commit/0697e6b688871c7a550bc2a856f72040dd2ed91b}{0697e6b688871c7a550bc2a856f72040dd2ed91b}
