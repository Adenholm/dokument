\section{\SNITFULL}


\subsection{Sammansättning}
\SNIT{} består av ordförande, vice ordförande, sekreterare och ledamöter.

\subsection{Åligganden}

\subsubsection{Det åligger \SNIT}

\begin{att}
	\item Administrera bevakning och utvärdering av utbildningskvalitén på kurser som ingår i IT-programmet.
	\item Årligen administrera nominering, omröstning och utdelning av sektionens pedagogiska pris.
	\item Lämna förslag, till sektionens valberedning, på representanter till nästkommande \SNIT.
\end{att}

\subsubsection{Det åligger \SNIT{}s ordförande}
\begin{att}
	\item Leda \SNIT{}s verksamhet.
	\item I studiefrågor representera sektionen och föra dess talan.
\end{att}

\subsubsection{Det åligger \SNIT{}s kassör}
\begin{att}
	\item Handha \SNIT{}s ekonomi.
\end{att}

\subsubsection{Det åligger \SNIT{}s vice ordförande}
\begin{att}
	\item Överta ordförandens åligganden i ordförandes frånvaro.
\end{att}

\subsubsection{Det åligger \SNIT{}s sekreterare}
\begin{att}
	\item Föra protokoll vid studienämndsmöte.
	\item Tillse att protokoll från studienämndsmöte anslås i enlighet med stadga.
	\item Tillse att material som inkommer till studienämnden anslås eller på annat sätt förmedlas dem det berör.
	\item Handha \SNIT{}s handlingar.
\end{att}

\subsection{Befogenheter}

\subsubsection{Det befogas \SNIT}
\begin{att}
	\item Utse studentrepresentanter ifrån olika studentgrupper inom utbildningsprogrammet Informationsteknik vid Chalmers, så som årskurser eller mastersprogram, vars uppgift är att representera snIT samt framför nämndens åsikter i studiefrågor.
\end{att}

\subsection{Mandatperiod}
Invalda till \SNIT{} betraktas som en del av \SNIT{} och sektionen från tillträdet, 1:a juli, till avträdet, 1:a juli ett år efter tillträdet.

Det yttersta ansvaret för \SNIT{}s verksamhet och ekonomi övergår till de nyinvalda den 1:a juli.
