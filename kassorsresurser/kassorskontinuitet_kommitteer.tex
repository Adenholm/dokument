\documentclass{article}
\usepackage[utf8]{inputenc}

\title{Kontinuitet kring händelser för specifika kommittéer}
\author{Katarina Bergbom \\
Reviderad av Erik Johnsson 10/2019}
\date{Februari 2018}


\usepackage{natbib}
\usepackage{graphicx}
\usepackage{hyperref}
\usepackage{pdfpages}


\newcommand\tab[1][1cm]{\hspace*{#1}}
\newcommand{\prit}{P.R.I.T. }
\newcommand{\prits}{P.R.I.T.s }
\newcommand{\fritid}{frITid }
\newcommand{\nollkit}{NollKIT }
\newcommand{\sexit}{sexIT }
\newcommand{\fanbarerit}{FanbärerIT }
\newcommand{\snit}{snIT }
\newcommand{\mrcit}{MRCIT }
\newcommand{\digit}{digIT }
\newcommand{\armit}{ArmIT }
\newcommand{\styrit}{StyrIT }
\newcommand{\styrits}{StyrITs }
\newcommand{\sunfleet}{Sunfleet }

\begin{document}

\maketitle

\tableofcontents

\section{\armit}
ArmIT får 30 000 kr i lån vid sin uppstart. \armit är också de som har företagskontakt, vilket innebär att om någon annan kommittée är intressearde av spons eller liknande så ska kontakten ske i samråd med \armit. Det grundar sig i att vi som sektion ska ha homogena prismallar till företag och verka som ett ansikte utåt, samt undvika dubbel kontakt och missförstånd. Vid betalningen är det \armit som skickar ut faktura och tar betalt, och sedan vidarebefordrar till berörd kommittée. 

\section{\digit}
\digit får bidrag från \styrit. Dels ett bidrag som går direkt till kommittéen, dels ett som går till musiktjänster som de handhar. Bidraget höjdes i budgeten 17/18 från 6000 kr till 12000kr per år. Dessa pengar går mycket till inventarier, som det tidigare hade behövt äskats för, vilket gör att jag rekommenderar att höjningen ligger kvar om det finns utrymme i budgeten för det. \digit bokför kostnaden på musiktjänsten även om den går på StyrITs budget. Hos \styrit bokförs enbart ett överlämnande av pengar. Det är också \digit som har ansvar för sektionens dns-addresser, dvs. chalmers.it och nollk.it. chalmers.it går på sektionens budget, och nollk.it går på NollKITs-budget. Dessa betalas direkt med kort, vilket görs direkt av den som ska betala, dvs \styrit och \nollkit. Då får \digit en betald-faktura, vilket de skickar till resp. kommittée och används vid bokföringen.

\section{\fanbarerit}
\fanbarerit får bidrag av \styrit. Storleken sätts i vår budget och bör samordnas, och betalas ut så fort budgeten har gått igenom. Tidigare har detta varit 5 000 kr, vilket har fungerat bra. \fanbarerit får också betalt för en frack/balklänningstvätt om de har genomfört minst ett arr under sitt år. Detta kostar ca. 1 000 kr styck, vilket förs över tillsammans med bidraget. Det som inte går åt, går tillbaka till \styrit. 

\section{\fritid}
Fritid beräknas inte gå med någon vinst, utan lever på bidrag från \styrit som betalas ut i klumpsumma när de går på. Vi betalar också lokalhyran till dem, och mitt år uppgick bidraget plus lokalhyra till totalt 14 000. De äskar dock i nuläget för många aktiviteter när deras pengar inte räcker till, varför denna summan skulle kunna höjas om det finns plats i budget. 

\section{\mrcit}
\mrcit får bidrag från masterprogrammen, detta måste godkännas år för år. Större delen av bidraget ges av Chalmers (SE och ID, sker i kommunikation med Jörgen). Detta bidrag måste godkännas av en central person, så viktigt att fakturer hamnar på rätt räkenskapsår för Chalmers, dvs fakturan för mottagningen under hösten 2019 ska in absolut senast sista november 2019. Där finns det 5 000 kr per program, samt ytterligare 100 kr per chalmersstudent att fakturera för. Fakturan ska mailas \textit{fakturaservice@chalmers.se}  och döpas till  \textit{Mastermottagningskostnader MPSOF och MPIDE}, med följande fakturaadress: \\

\textit{
\tab[0.3cm] Fakturaservice\\
\tab Chalmers Tekniska Högskola AB\\
\tab Jörgen Blennow/KST 98300  \\
\tab 412 96 Göteborg} \\

Sedan kan också GU faktureras för GU-studenterna. De kan faktureras för 100 kr per student, till addressen nedan:  \\

\textit{
\tab[0.3cm] Fakturaservice\\
\tab Chalmers Tekniska Högskola AB\\
\tab Peter Ljunglöf / kst 3710 \\
\tab 412 96 Göteborg}

\section{\nollkit}
\nollkit får bidrag, nuvarande 115 000 kr av PL, som går till hela deras verksamhet. De fakturerar PL med sitt egna bankkontonummer, så att pengarna inte går via \styrit. I och med att NollKIT främst lever på bidrag från PL ska de gå plus-minus-noll under sitt verksamhetsår. Utöver PL-pengarna får de ett uppstartslån på 20 000 kr av StyrIT, att använda innan PL-pengarna kommer in. Lånet ska betals tillbaka i hela summan. \nollkit betalar också nollk.it själva, även om det är \digit som har hand om domänen. Står hur betalningen genomförs under deras sektion. 

\section{\prit}
\prit får ett lån av \styrit när de går på, som är ungefär 30 000 kr då de har utgifter innan de får in pengar från sin första pubrunda. \prit har också hand om inköp av förbrukningsvaror och hubbenrust (utvecklingskostnader Hubben), vilket de får bidrag av StyrIT från. Dessa bidrag överförs halvårsvis i klumpsummor, då \styrit och \prit har olika budgetår. Det är summan i \styrits budget som gäller vid utbetalningarna, så kassören i \prit bör kolla dessa summor innan de sätter sin budget vid vintern. I och med att \prits främsta rustarr är rustveckan på sommaren, så bör rustbidraget vara ca 5 000 kr större under våren. Förbrukningsvaror bör betalas ut jämt för våren och hösten. Skulle det vara att allt bidrag inte har gått åt på vintern, så ligger dessa kvar på \prits konto. Däremot så görs avstämningen innan årsbokslutet vid sommaren och eventuellt outnyttjat bidrag betalas då tillbaka. \prit siktar på att gå +-0, men går de plus betalas resultatet tillbaka till \styrit. 

\section{\sexit}
\sexit får ett uppstartslån, nuvarande på 30 000 kr. De beräknas för närvarande inte att gå plus, då resultatet har varit varierande de senaste åren. \sexit arrangerar Pedagogiska Priset, tillsammans med \snit. Där ger \snit ett bidrag som går till fasta kostnader såsom lokalhyra, samt köper ut biljetter som de delar ut till nominerade. Annars så håller \sexit i budgeten och sittningen, säljer biljetter med mera, och får eventuell vinst. \sexit hyr också Gasquen. När det görs, på liknande sätt som att hyra bulten, så får de sedan två fakturor, en för kostnaderna och en för inkomsten. 

\section{\snit}
\snit får sina pengar från PL, nuvarande 40 000 kr. 30 000 kr av dessa är en klumpsumma för studiefrämjande ändamål, vilket är hela deras verksamhet, dvs. profilering och liknande går också under detta. 5 000 kr går till Pedagogiska Priset-sittningen, dels lokalhyra och fasta kostnader samt biljetter för de nominerade. De får också 5 000 kr som delas ut i pris till vinnaren. \snit ska gå plus minus noll, och i och med att deras pengar är öronmärkta så ligger de kvar på deras konto om de inte skulle gå åt. \snit fakturerar PL direkt med sitt egna kontonummer, så att det inte behöver gå via \styrit. Maximala matkostnaden som får läggas på studenter gäller inte \snit, det finns undantag i stadgan för detta. Anledning är att det bjuds på mat på kursrepresantsmöten, vilket går över totalsumman oftast. 

\section{Hyrning av Bulten}
När man ska hyra Bulten går det till enligt följande: 
\begin{enumerate}
    \item Prata med bultenstyret/lokalbokning för att boka hubben. Man måste ha gått SUS för att få vara serveringsansvarig.
    \item Prata med vSO (vso@chalmersstudentkar.se). Nämn att ni \textbf{behöver en extra kassa}, annars kommer ni enbart kunna ta betalt med kårkort!
    \item Ni ska få pengarna direkt till kommittéen, men även här prata med vSO så att hon vet vem som ska ha dem. 
\end{enumerate}

\section{Adresser}
Det finns en del olika adresser att hålla koll på, både vid fakturering och skickande till sektionen och från sektionen. Följande är vad som gäller: 

\subsection{Adressering vid brev till sektionen}
Notera att brev är sådant som går ner i ett brevinkast hemma. Större saker är inte brev. Brev skickas ner till kårhuset, enl. följande adress: \\

\textit{\tab[0.5cm]Chalmers Studentkår \\
\tab Teknologsektionen Informationsteknik \\
\tab StyrIT (För kommittéer gäller era namn) \\
\tab Kontaktperson \\
\tab Teknologgården 2 \\
\tab 412 58 Göteborg}

\subsection{Adressering vid paket till sektionen}
För paket eller ännu större saker, ex. försändelser på pall, gäller annan adress. Den adressen är adressen till dit sakerna skall och framför allt där du som ansvarig mottagare kommer att vara när leveransen kommer. Det skall alltså \textbf{inte} skickas till Kårhuset. Chalmers har en godsmottagning man kan prata med, men för mindre paket så är det enklast att levera till mottagarpersonens adress. Typ din. 

\subsection{Fakturaadress till PL}
\label{sec:pl-fakturaadress}
Fakturaadressen till PL är enligt nedan. Fakturor till PL ska också ha kostnadsställe 98302 och projektnummer 991203 \\

\textit{
\tab[0.3cm]Helena Skoglund \\
\tab Chalmers Tekniska Högskola \\
\tab Fakturaservice\\
\tab 412 96 Göteborg}

\section{Inköp av pins/profilering som ska säljas i flera år framåt}


\end{document}
