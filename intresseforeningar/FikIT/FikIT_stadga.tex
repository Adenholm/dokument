\documentclass[11pt, noincludeaddress]{classes/cthit}
\usepackage{titlesec}
\usepackage{titletoc}
\usepackage{verbatimbox}

\titleformat{\paragraph}[hang]{\normalfont\normalsize\bfseries}{\theparagraph}{1em}{}
\titlespacing*{\paragraph}{0pt}{3.25ex plus 1ex minus 0.2ex}{0.7em}


\titlelabel{\S \thetitle\quad}

\graphicspath{ {images/} }

\begin{document}

\title{FikITs Stadga}
\approved{2016--12--16}
\revisioned{2020--04--02}
\maketitle

\thispagestyle{empty}

\newpage

\makeheadfoot%

%Rubriksnivådjup
\setcounter{tocdepth}{2}
%Sidnumreringsstart
\setcounter{page}{1}
\tableofcontents

\newpage

\section{Definition}




\subsection{Benämning}

\subsubsection{Föreningens fullständiga namn}
\FIKITFULL{}

\subsubsection{Föreningens akronym}
\FIKIT{}




\subsection{Syfte}

\subsubsection{Främjande av intresse}
Föreningen har som syfte att värna om IT-teknologernas fika- och bakintresse.

\subsection{Rättigheter}
\subsubsection{Sektionens varumärke}
Föreningen äger rätt att i namn och emblem använda sektionens namn och symboler. 


\subsection{Skyldigheter}

\subsubsection{Sektionen}
Föreningen är skyldig att rätta sig efter sektionens stadga, reglemente, övriga styrdokument och fattade beslut. 

\subsubsection{Styrelse}
Intresseföreningen måste ha en tillsatt styrelse, se § \ref{styrelse}. 

\subsubsection{Årsmöten}
Intresseföreningen måste ha minst ett möte per år dit föreningens medlemmar är kallade. 

\subsection{Verksamhet}
\subsubsection{Ekonomi}
Föreningens ekonomi skall vara fristående från sektionen.

\subsubsection{Revision}
Föreningens verksamhet och ekonomi kan komma att granskas av sektionens revisorer.

\newpage

\section{Medlemmar}

\subsection{Styrelse}
\label{styrelse}

\subsubsection{Sammansättning}
Intresseföreningens styrelse består av: 
\begin{itemize}
        \item Ordförande
		\item Kassör
		\item upp till sex övriga ledamöter
\end{itemize}

\subsubsection{Det åligger föreningens ordförande att:}
Leda verksamheten, handha FikITs handlingar och teckna FikITs firma.

\subsubsection{Det åligger föreningens kassör att:}
Hantera föreningens ekonomi. 



\subsubsection{Ansvar}
Styrelsen ansvarar för intresseföreningens medlemslista, medlemsvärvning, sammankomster, att beslut som tas på årsmötet genomförs och övrig verksamhet.

\subsubsection{Tillsättning}
Styrelsen tillsätts av årsmötet och tillträder direkt efter valet. Valbar är medlem i föreningen. En medlem kan inte inneha mer än en post i styrelsen.


\subsection{Medlemskap}

\subsubsection{Medlemmar}
Som medlem antas intresserad som godkänner dessa stadgar och aktivt tar ställning. Medlemskap fås genom att årligen göra en skriftlig anmälan till föreningen. 

\subsubsection{Sektionsmedlemmar}
Varje medlem i Teknologsektionen Informationsteknik har rätt till medlemskap i intresseföreningen.

\subsubsection{Övrig medlem}
Intresseföreningens styrelse har rätt att besluta om medlemskap för övriga.

\subsubsection{Uteslutande}
Medlem som motverkar föreningens syfte, eller inte anses bidra till intresseföreningens arbete kan uteslutas genom beslut av intresseföreningens styrelse. 

\subsection{Medlemsrättigheter}

\subsubsection{Närvaro-, yttrande- och förslagsrätt}
Föreningens revisorer samt varje medlem av IT-sektionen och/eller föreningen har närvarorätt, yttranderätt och förslagsrätt på föreningens samtliga årsmöten. 

\subsubsection{Rösträtt}
Varje medlem av intresseföreningen har rösträtt i föreningens samtliga årsmöten.





\section{Sammankomster}

\subsection{Årsmöte}

\subsubsection{Kallelse}
Intresseföreningens styrelse beslutar om tid och plats för föreningens årsmöte. 

\subsubsection{Mötets behörighet}
För att vara behörigt måste årsmötet utlysas minst tio dagar i förväg och referens till dagordningen finnas tillgänglig via sektionens anslagstavla, sektionens hemsida samt föreningens eventuella maillista. 

\subsubsection{Dagordning}
Följande punkter måste behandlas på det ordinarie mötet: 

\begin{itemize}
    \item Mötets öppnande 
    \item Mötets behörighet 
    \item Fastställande av dagordningen 
    \item Val av mötets ordförande 
    \item Val av mötets sekreterare 
    \item Val av två rösträknare tillika justerare 
    \item Verksamhetsberättelse 
    \item Ekonomisk berättelse 
    \item Revisorernas berättelse 
    \item Ansvarsfrihet för förra årets styrelse 
    \item Fikapaus 
    \item Verksamhetsplan 
    \item Budget 
    \item Motioner 
    \item Val av styrelse 
    \item Val av revisor 
    \item Övriga frågor
    \item Mötets avslutande 
\end{itemize}

Följande punkter måste behandlas på ett extrainsatt årsmöte: 
\begin{itemize}
    \item Mötets öppnande 
    \item Mötets behörighet 
    \item Val av mötets ordförande 
    \item Val av mötets sekreterare 
    \item Fikapaus 
    \item Övriga frågor 
    \item Mötets avslutande 
\end{itemize}

\subsubsection{Extra årsmöte}
Styrelsen skall kalla till ett extra årsmöte om detta krävs av antingen: föreningens revisorer, hälften av föreningens medlemmar eller styrelsen själv. 

\subsection{Beslut}

\subsubsection{Röstning}
Beslut genom röstning sker genom enkel majoritet. Vid lika röstning görs en ny röstning, vid fortsatt lika röstetal har föreningsordförande utslagsröst utom vid personval då lotten avgör. 

\subsubsection{Stadgeändring}
Ändringar av denna stadga kan enbart ske på ett årsmöte, varvid sektionsstyrelsen ska underrättas om justeringarna efter avslutat årsmöte. 

\end{document}
