\documentclass[11pt, noincludeaddress]{classes/cthit}
\usepackage{titlesec}
\usepackage{titletoc}
\usepackage{verbatimbox}

\titleformat{\paragraph}[hang]{\normalfont\normalsize\bfseries}{\theparagraph}{1em}{}
\titlespacing*{\paragraph}{0pt}{3.25ex plus 1ex minus 0.2ex}{0.7em}


\titlelabel{\S \thetitle\quad}

\graphicspath{ {images/} }

\begin{document}

\title{FlashITs stadga}
\approved{2003--01--27}
\revisioned{2014--11--04}
\maketitle

\thispagestyle{empty}

\newpage

\makeheadfoot%

%Rubriksnivådjup
\setcounter{tocdepth}{2}
%Sidnumreringsstart
\setcounter{page}{1}
\tableofcontents

\newpage

\section{Definition}

\subsection{Benämning}

\subsubsection{Föreningens fullständiga namn}
\FLASHITFULL{}

\subsubsection{Föreningens akronym}
\FLASHIT{}

\subsection{Syfte}

\subsubsection{Främjande av intresse}
Intresseföreningen har som syfte att värna om IT-teknologernas film- och fotointresse, samt att dokumentera sektionens verksamhet i bild och film i viss utsträckning.

\subsection{Skyldigheter}

\subsubsection{Sektionen}
Intresseförening är skyldig att rätta sig efter sektionens stadga, 
reglemente och fattade beslut. 

\subsubsection{Styrelse}
Intresseföreningen måste ha en tillsatt styrelse, se {styrelse}.

\subsubsection{Årsmöten}
Intresseföreningen måste ha minst ett möte per år dit föreningens 
och sektionens medlemmar är kallade.

\newpage

\section{Medlemmar}

\subsection{Styrelse}
\label{styrelse}

\subsubsection{Sammansättning}
Intresseföreningens styrelse består av ordförande, samt upp till 7
övriga ledamöter.

\subsubsection{Styrelsen åligganden}
Det åligger styrelsen:

\begin{att}
        \item Anordna ett arrangemang för sektionens medlemmar per läsperiod
        \item Handha FlashITs ekonomi
\end{att}

\subsubsection{Ordförandes åligganden}
Det åligger styrelsens ordförande:

\begin{att}
        \item Leda verksamheten
        \item Handha FlashITs handlingar
        \item Teckna FlashITs firma
\end{att}

\subsubsection{Ansvar}
Styrelsen ansvarar för intresseföreningens medlemslista, 
medlemsvärvning, sammankomster, beslut som tas på årsmötet och 
övrig verksamhet.

\subsubsection{Tillsättning}
Styrelsen tillsätts av årsmötet och tillträder direkt efter valet. Valbar är medlem i föreningen. En medlem kan inte inneha mer än en post i styrelsen.

\subsection{Medlemskap}

\subsubsection{Medlemmar}
Som medlem antas intresserad som godkänner dessa stadgar och aktivt tar ställning. Medlemskap fås genom att årligen göra en skriftlig anmälan till föreningen.

\subsubsection{Sektionmedlemmar}
Varje medlem i teknologsektionen Informationsteknik har rätt till 
medlemskap i intresseföreningen.

\subsubsection{Övrig medlem}
Intresseföreningens styrelse har rätt att besluta om medlemskap för 
övriga.

\subsubsection{Uteslutande}
Medlem som motverkar föreningens syfte, eller inte anses bidra till 
intresseföreningens arbete kan uteslutas genom beslut av 
intresseföreningens styrelse.

\subsection{Medlemsrättigheter}

\subsubsection{Närvaro-, yttrande- och förslagsrätt}
Varje medlem av intresseföreningen har närvarorätt, yttranderätt och 
förslagsrätt på föreningens samtliga årsmöten.

\subsubsection{Rösträtt}
Varje medlem av intresseföreningen har rösträtt i föreningens 
samtliga årsmöten.

\newpage

\section{Sammankomster}

\subsection{Årsmöte}

\subsubsection{Kallelse}
Intresseföreningens styrelse beslutar om tid och plats för föreningens 
årsmöte. Årsmötet skall utlysas minst fem läsdagar i förväg på 
sektionens anslagstavla.

\subsubsection{Dagordning}
Följande punkter måste behandlas på det ordinarie mötet:

\begin{itemize}
        \item Mötets öppnande
        \item Mötets behörighet
        \item Val av mötets ordförande
        \item Val av mötets sekreterare
        \item Verksamhetsberättelse
        \item Ekonomisk berättelse
        \item Ansvarsfrihet för förra årets styrelse
        \item Val av styrelse
        \item Övriga frågor
        \item Mötets avslutande
\end{itemize}

Följande punkter måste behandlas på ett extrainsatt möte:

\begin{itemize}
        \item Mötets öppnande
        \item Mötets behörighet
        \item Val av mötets ordförande
        \item Val av mötets sekreterare
        \item Övriga frågor
        \item Mötets avslutande
\end{itemize}

\subsubsection{Extra årsmöte}
Om styrelsen anser det nödvändigt eller minst hälften av föreningens medlemmar 
kräver det så ska styrelsen eller sektionsstyrelsen kalla till ett extra årsmöte.

\subsection{Beslut}

\subsubsection{Röstning}
Beslut genom röstning sker genom enkel majoritet.

\subsubsection{Stadgeändring}
Denna stadga kan bara ändras på ett årsmöte där sektionsstyrelsen ska underrättas om justeringarna efter avslutat årsmöte.

\end{document}
