\section{Allmänt}

\subsection{Definition}

\subsubsection{Föreningsform}
Teknologsektionen Informationsteknik, härmed benämnd teknologsektionen, är en ideell förening vilken består av studerande vid utbildningsprogrammet Informationsteknik vid Chalmers Tekniska Högskola. Teknologsektionen är fackligt och partipolitiskt obunden samt religiöst neutral.

\subsubsection{Uppdrag}
Teknologsektionen har som uppdrag att föra studiebevakning för
sektionens medlemmar och verka för att varje medlem skall kunna tillgodogöra
sin utbildning. Teknologsektionen ska också verka för att varje medlem ska vara
väl förberedd för arbetslivet, vara psykosocialt och fysiskt trygg samt känna en
studiesocial gemenskap under sin tid som teknolog.

\subsection{Medlemmar}

Medlem i teknologsektionen är den som är eller har varit inskriven vid utbildningsprogrammet Informationsteknik vid Chalmers och betalar sektionsavgift. Medlem är även studerande vid Chalmers efter erlagd sektionsavgift och godkännande från teknologsektionens styrelse. Därutöver kan teknologsektionen utse hedersmedlemmar, enligt \paragraphref{sec:hedersmedlemmar}.

\subsection{Verksamhetsår}
Teknologsektionens verksamhetsår löper från och med 1:a juli till och med 30:e juni påföljande år.
