\section{Sektionskommittéer}

\subsection{Definition}

\subsubsection{Sammansättning}
Sektionskommitté skall ha ordförande, kassör samt ett i reglemente fastställt minst antal förtroendeposter.

\subsubsection{Uppgift}
Sektionskommitté skall verka för sektionens bästa och ha en i reglemente fastslagen uppgift.

\subsubsection{Myndighet}
Ordförande och kassör i sektionskommittéen skall vara myndiga.

\subsection{Förteckning}

\subsubsection{Förteckning}
Sektionskommittéer är de i reglemente förtecknade.

\subsection{Rättigheter}

\subsubsection{Rättigheter}
Sektionskommitté äger rätt att i namn och emblem använda sektionens namn och symboler.

\subsection{Skyldigheter}

\subsubsection{Skyldigheter}
Sektionskommitté är skyldig att rätta sig efter sektionens stadga, reglemente och fattade beslut.

\subsection{Ekonomi}

\subsubsection{Föreningens firma}
Ordförande och kassör för en sektionskommitté tecknar var för sig
föreningens firma.

\subsubsection{Verksamhet och revision}
Föreningens verksamhet och ekonomi granskas av sektionens revisorer.
