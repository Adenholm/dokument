\section{Studienämnden, \SNIT}

\subsection{Definition}

\subsubsection{Sammansättning}
Studienämnden Informationsteknik består av ordförande, vice ordförande, kassör samt i reglemente fastställt antal förtroendeposter. Ordförande och kassör i studienämnden skall vara myndiga.

\subsubsection{Uppdrag}
Studienämnden Informationsteknik, \SNIT{}, har till uppdrag att inom teknologsektionen övervaka studiefrågor, aktivt verka för bra/bättre kurser, främja kontakten med examinatorer samt hålla god kontakt med teknologsektionens medlemmar.

\subsection{Protokoll}
Protokoll skall föras vid studienämndsmöte. Protokollet skall justeras av en studienämndsmedlem och korrekt anslås, enligt \paragraphref{sec:protokoll:anslagning}, senast tio läsdagar efter studienämndsmötet.

\subsection{Rättigheter}
Studienämnd äger rätt att i namn och emblem använda teknologsektionens namn och dess symboler.

\subsection{Skyldigheter}
Studienämnden är skyldig att rätta sig efter teknologsektionens stadga, reglemente och fattade beslut.

\subsection{Ekonomi}
\subsubsection{Ansvar}
Ordförande och kassör är gemensamt ansvariga för studienämndens ekonomi.
\subsubsection{Föreningens firma}
Ordförande och kassör för studienämnden tecknar var för sig föreningens firma.
\subsubsection{Verksamhet och revision}
Studienämndens verksamhet och ekonomi granskas av teknologsektionens revisorer.