\section{Ändrings- och tolkningsfrågor}

\subsection{Stadgeändring}

\subsubsection{Förutsättningar}
Ändring av stadgan kan endast göras av sektionsmötet med 2/3 majoritet vid två på varandra följande sektionsmöten, varav minst ett ordinarie, med minst tio läsdagars mellanrum.

\subsubsection{Kårstyrelsens godkännande}
Ändring av eller tillägg till stadgan skall godkännas av kårstyrelsen.
\subsubsection{Dokumentation}
Ändring av stadgan skall dokumenteras. Dokumentationen skall innehålla:
\begin{itemize}
	\item datum och sektionsmöte
	\item den tidigare formuleringen
\end{itemize}

\subsection{Reglementesändring}
Ändring av eller tillägg till reglementet kan göras av sektionsmötet med 2/3 majoritet. Ändringen träder i kraft efter att sektionsmötet avslutats.

\subsection{Ändring av policies och övriga styrdokument}
Ändring av eller tillägg till policies och övriga styrdokument kan göras av sektionsmötet. Ändringen träder i kraft efter att sektionsmötet avslutats.

\subsection{Tolkningstvist}

\subsubsection{Stadgans tolkning}
Ändring av eller tillägg till styrdokumentet kan göras av sektionsmötet. Ändringen träder i kraft efter att sektionsmötet avslutats.

\subsubsection{Reglementets tolkning}
Vid tolkning av reglemente gäller tills frågan avgjorts av sektionsmötet,
sektionsstyrelsens mening.

\subsubsection{Policies och övriga styrdokuments tolkning}
Vid tolkning av policies och övriga styrdokument gäller tills frågan avgjorts av sektionsmötet, sektionsstyrelsens mening.