\section{Ändrings- och tolkningsfrågor}

\subsection{Stadgeändring}

\subsubsection{Förutsättningar}
Ändring av stadgan kan endast göras av sektionsmötet med 2/3 majoritet vid två på varandra följande sektionsmöten, varav minst ett ordinarie, med minst tio läsdagars mellanrum.

\subsubsection{Undantag}
Redaktionella ändringar får göras av sektionsstyrelsen. Redaktionella ändringar måste presenteras på nästkommande sektionsmöte efter deras införande.

\subsubsection{Kårstyrelsens godkännande}
Ändring av eller tillägg till stadgan skall godkännas av kårstyrelsen.
\subsubsection{Dokumentation}
Ändring av stadgan skall dokumenteras. Dokumentationen skall innehålla:
\begin{itemize}
	\item datum och sektionsmöte
	\item den tidigare formuleringen
\end{itemize}

\subsection{Reglementesändring}
\subsubsection{Förutsättningar}
Ändring av eller tillägg till reglementet kan göras av sektionsmötet med 2/3 majoritet. Ändringen träder i kraft efter att sektionsmötet avslutats.

\subsubsection{Undantag}
Redaktionella ändringar får göras av sektionsstyrelsen. Redaktionella ändringar måste presenteras på nästkommande sektionsmöte efter deras införande.

\subsection{Ändring av policies och övriga styrdokument}
\subsubsection{Förutsättningar}
Ändring av eller tillägg till policies och övriga styrdokument kan göras av sektionsmötet. Ändringen träder i kraft efter att sektionsmötet avslutats.

\subsubsection{Undantag}
Redaktionella ändringar får göras av sektionsstyrelsen. Redaktionella ändringar måste presenteras på nästkommande sektionsmöte efter deras införande.

\subsection{Tolkningstvist}
Vid tolkning av styrdokument gäller sektionsstyrelsens mening, tills att frågan avgjorts av sektionsmötet.
