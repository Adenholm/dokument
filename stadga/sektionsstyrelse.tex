\section{\STYRITFULL}

\subsection{Uppdrag}
Sektionsstyrelsen handhar den verkställande ledningen av teknologsektionens verksamhet i överrensstämmelse med denna stadga, befintligt reglemente, policies, övriga styrdokument samt av sektionsmötet fattade beslut.

\subsection{Sammansättning}
Sektionsstyrelsen består av ordförande, vice ordförande, kassör samt i reglemente fastställt antal ledamöter. Ordförande och kassör i styrelsen skall vara myndiga.

\subsection{Rättigheter}
Sektionsstyrelsen äger rätt att i namn och emblem använda sektionens namn och dess symboler.

\subsection{Ansvar}
Sektionsstyrelsen ansvarar inför sektionsmötet för sektionens verksamhet och ekonomi.

\subsection{Firmateckning}
Ordförande och kassör i sektionsstyrelsen tecknar sektionens firma var för sig.

\subsection{Styrelsemöte}

\subsubsection{Sammanträden}
Sektionsstyrelsen sammanträder minst tre gånger per läsperiod.

\subsubsection{Kallelse}
Sektionsstyrelsen sammanträder på kallelse av styrelsens ordförande eller vice ordförande. Styrelsemedlem äger rätt att hos ordförande eller vice ordförande begära utlysande av styrelsemöte. Sådant möte skall hållas inom 5 läsdagar.

\subsubsection{Beslutsmässighet}
Styrelsemötet är beslutsmässigt när minst hälften av medlemmarna är närvarande. Dessutom måste ordförande eller vice ordförande vara närvarande.

\subsubsection{Protokoll}
Protokoll skall föras vid styrelsemöte. Protokollet skall justeras av två medlemmar av sektionsstyrelsen och anslås korrekt, enligt \paragraphref{sec:protokoll:anslagning}, senast tio läsdagar efter styrelsemötet.

\subsubsection{Fyllnadsval}
Styrelsemötet har befogenhet att hålla fyllnadsval av ledarmöter till studienämnden, sektionsfunktionärer eller sektionskommittéer i samråd med valberedningen och medlemmarna i det relevanta organet. Notera att detta exkluderar styrelsen.

\subsection{Avsättning}

\subsubsection{Förutsättningar}
För att avsätta sektionsstyrelsen krävs att ärendet är korrekt anslaget enligt \paragraphref{sec:protokoll:anslagning} senast tre läsdagar innan sektionsmöte, samt att mötet är beslutsmässigt och minst 3/4 av de röstberättigade vid mötet är ense om beslutet. 

\subsubsection{Förfarande}
Vid sådant möte skall interimsstyrelse och ny valberedning väljas. Interimsstyrelsen utfärdar kallelse till extra sektionsmöte där ny ordinarie styrelse skall väljas. Detta sektionsmöte skall hållas inom 15 läsdagar från det sektionsmöte då interimsstyrelsen valdes och under ordinarie terminstid.

\subsubsection{Interimsstyrelsens befogenheter och skyldigheter}
Interimsstyrelsen övertar ordinarie styrelsens befogenheter och skyldigheter tills ny ordinarie sektionsstyrelse valts.


