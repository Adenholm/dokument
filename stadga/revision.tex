\section{Revision och ansvarsfrihet}

\subsection{Revisorer}

\subsubsection{Inval}
Sektionsmötet utser två till fyra lekmannarevisorer med uppgift att granska teknologsektionens verksamhet och ekonomi under verksamhetsåret.

\subsubsection{Förutsättningar}
Teknologsektionens revisorer ska 
\begin{itemize}
    \item ej inneha en ansvarspost på sektionen under sitt verksamhetsår.
    \item vara ansvarsbefriade från eventuella ansvarsuppdrag inom teknologsektionen. 
\end{itemize} 

Med ansvarspost menas innehav av posten ordförande eller kassör i någon av teknologsektionens kommittéer, nämnder eller styrelser.

\subsubsection{Revision}
Räkenskaper och övriga handlingar skall vara revisorerna tillhanda senast 15 läsdagar före ordinarie sektionsmöte. 

Sektionens revisorer har dessutom en rättighet att med en notis på tre läsdagar få handlingar och räkenskaper tillhanda. 

\subsection{Åligganden}

\subsubsection{Åligganden}
Det åligger revisorerna att korrekt anslå, enligt \paragraphref{sec:protokoll:anslagning} revisionsberättelser senast tre läsdagar före ordinarie sektionsmöte.

\subsubsection{Revisionsberättelsen}
Revisionsberättelsen skall innehålla yttrande om ansvarsfrihet för berörda personer.

\subsection{Ansvarsfrihet}

\subsubsection{Beslut}
Ansvarsfrihet är beviljad berörda personer då sektionsmötet fattat beslut om detta.

\subsubsection{Undantag}
Skulle förtroendevald på sektionen med ekonomiskt ansvar avgå före mandatperiodens slut, skall revision företagas.