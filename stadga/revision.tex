\section{Revision och ansvarsfrihet}

\subsection{Revisorer}

\subsubsection{Inval}
Sektionsmötet utser två till fyra lekmannarevisorer med uppgift att granska sektionens verksamhet och ekonomi under verksamhetsåret.

\subsubsection{Förutsättningar}
Sektionens revisorer kan ej inneha posten som ordförande eller kassör i sektionsstyrelse, studienämnd eller sektionskommittéer på sektionen under sitt verksamhetsår. Sektionens revisorer ska vara ansvarsbefriade från eventuella ansvarsuppdrag inom sektionen. Med ansvarsuppdrag menas innehav av posten ordförande eller kassör i någon av sektionens kommittéer, nämnder eller styrelser.

\subsubsection{Revision}
Räkenskaper och övriga handlingar skall vara revisorerna tillhanda senast 15 läsdagar före sektionsmöte.

\subsection{Åligganden}

\subsubsection{Åligganden}
Det åligger revisorerna att korrekt anslå, enligt \ref{sec:protokoll:anslagning}revisionsberättelser senast tre läsdagar före ordinarie sektionsmöte.

\subsubsection{Revisionsberättelsen}
Revisionsberättelsen skall innehålla yttrande ifråga om ansvarsfrihet för berörda personer.

\subsection{Ansvarsfrihet}

\subsubsection{Sektionsmötets beslut}
Ansvarsfrihet är beviljad berörda personer då sektionsmötet fattat beslut om detta.

\subsubsection{Undantag}
Skulle förtroendevald på sektionen med ekonomiskt ansvar avgå före mandatperiodens slut, skall revision företagas.