\section{Intresseföreningar}

\subsection{Definition}

\subsubsection{Sammansättning}
Intresseförening är en sammanslutning medlemmar med ett gemensamt intresse. Intresseföreningen ska ha en styrelse bestående av föreningens medlemmar. Minst hälften av medlemmarna i intresseföreningens styrelse ska bestå av sektionsmedlemmar.

\subsubsection{Medlemsrätt}
Varje sektionsmedlem skall ha rätt till medlemskap. Föreningsmedlem som motverkar föreningens syften kan dock uteslutas. Styrelsen kan även besluta om medlemskap för de som inte är medlemmar av sektionen.

\subsubsection{Stadga}
Intresseförening skall ha en av sektionsstyrelsen godkänd stadga.

\subsubsection{Uppgift}
Intresseförening skall verka för sektionens bästa och ha en i reglemente fastslagen uppgift.

\subsection{Förteckning}

\subsubsection{Förteckning}
Sektionens intresseföreningar är de i reglemente förtecknade.

\subsection{Rättigheter}

\subsubsection{Rättigheter}
Intresseförening äger rätt att i namn och emblem använda sektionens namn och symboler.

\subsection{Skyldigheter}

\subsubsection{Skyldigheter}
Intresseförening är skyldig att rätta sig efter sektionens stadga, reglemente och fattade beslut.

\subsection{Ekonomi}

\subsubsection{Ekonomi}
Intresseförening skall ha en fristående ekonomi.

\subsubsection{Verksamhet och revision}
Sektionens revisorer har rätt att granska föreningens verksamhet och ekonomi.