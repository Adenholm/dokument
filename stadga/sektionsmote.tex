\section{Sektionsmötet}

\subsection{Befogenheter}
Sektionsmötet är teknologsektionens högsta beslutande organ.

\subsection{Sammanträden}
Det skall hållas fyra ordinarie sektionsmöten varje år, ett per läsperiod. Utöver detta kan extra sektionsmöten hållas.

\subsection{Utlysande}
\label{sec:utlysande}

\subsubsection{Kallelse}
Sektionsmötet sammanträder på kallelse av sektionsstyrelsen.

\subsubsection{Utlysningsrätt}
Rätt att hos sektionsstyrelsen begära utlysande av sektionsmöte tillkommer:

\begin{itemize}
	\item medlem i styrelsen
	\item kårens inspektor
	\item kårstyrelsen
	\item teknologsektionens inspektor
	\item teknologsektionens revisorer
	\item sektionsmedlem, förutsatt att minst 25 sektionsmedlemmar stödjer förslaget
\end{itemize}

Efter begäran om utlysning av sektionsmöte ska utlysandet ske inom 5 läsdagar. Efter utlysning skall sektionsmötet hållas inom tolv läsdagar

\subsubsection{Utlysning}
Sektionsmötet skall utlysas minst tio läsdagar i förväg genom att kallelse enligt reglemente anslås. Inkomna motioner och propositioner skall anslås minst tre läsdagar i förväg. Övriga möteshandlingar skall anslås enligt stadga eller reglemente. 

\subsection{Åligganden}
Sektionsmötet har följande åligganden:
\subsubsection{Första ordinarie höstmötet}
\begin{itemize}
	\item Val av: 
	\begin{itemize}
		\item valberedning
		\item revisorer
	\end{itemize}
	\item Behandlande av års- och revisionsberättelse och ansvarsfrihet för:
	\begin{itemize}
		\item sektionsstyrelse
		\item studienämnd
	\end{itemize}
\end{itemize}
	
\subsubsection{Första ordinarie vårmötet}
\begin{itemize}
	\item Val av: 
	\begin{itemize}
		\item inspektor (udda kalenderår)
	\end{itemize}
\end{itemize}


\subsubsection{Andra ordinarie vårmötet}
\begin{itemize}
	\item val av: 
	\begin{itemize}
		\item sektionsstyrelse
		\item studienämnd
	\end{itemize}
	\item Fastställande av preliminär budget och verksamhetsplan för teknologsektionen.
	\item Fastslående av sektionsavgift för nästkommande läsår.
\end{itemize}

\subsubsection{Sektionskomittéer}
\begin{itemize}
	\item Val av sektionskommittémedlemmar på det sektionsmöte reglementet anger för respektive kommitté. 
	\item Fastställande av budget och verksamhetsplan på det sektionsmöte reglementet anger för respektive kommitté.
	\item Behandlande av ansvarsfrihet för sektionskommittéer.
\end{itemize}

\subsection{Beslutförhet}

\subsubsection{Beslutsmässighet}
Sektionsmötet är beslutsmässigt om mötet är behörigt utlyst enligt \ref{sec:utlysande}

\subsubsection{Krav}
Beslut kräver att minst x medlemmar är närvarande. \\
Där x utgör $min(0.10 * totala\_antalet\_medlemmar, 30)$. \\
Vid färre medlemmar bordläggs beslutsfattningen till nästkommande möte.

\subsection{Motion}
Medlem som önskar ta upp motion på sektionsmötet skall tillhandahålla motionen till sektionsstyrelsen senast sju läsdagar före sektionsmötet.

\subsection{Överklagande}
Beslut av sektionsmötet som strider mot kårens eller sektions stadga, reglemente eller policies får undanröjas av kårstyrelsen. Sådant beslut kan tas upp till prövning om det begärs av en kårmedlem då det rör kårens stadga eller en sektionsmedlem då det rör teknologsektionens stadga.

\subsection{Omröstning}

\subsubsection{Fullmakt}
Röstning med fullmakt får ej ske.

\subsubsection{Omröstningsförfarande}
Omröstning skall ske öppet, om ej sluten votering begärs. Vid lika röstetal har mötesordföranden utslagsröst utom vid personval då lotten avgör.

\subsubsection{Beslut}
Om inget annat anges antas samtliga beslut med enkel majoritet.

\subsection{Närvaro-, yttrande-, förslags- och rösträtt}

\subsubsection{Närvaro- och yttranderätt tillkommer:}
\begin{itemize}
	\item sektionsmedlem
	\item hedersmedlem
	\item kårledningsmedlemmar
	\item kårens inspektor
	\item teknologsektionens revisorer
	\item teknologsektionens inspektor
	\item av mötet adjungerade ickemedlemmar
\end{itemize}

\subsubsection{Förslags- och rösträtt}
Förslags- och rösträtt tillkommer endast sektionsmedlemmar.

\subsection{Protokoll}
Sektionsmötesprotokoll skall justeras av två av mötet valda justeringspersoner. Justerat protokoll skall korrekt anslås, enligt \ref{sec:protokoll:anslagning}, senast tio läsdagar efter mötet.
