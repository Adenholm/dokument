\section{Sektionsmötet}

\subsection{Befogenheter}

\subsubsection{Befogenheter}
Sektionsmötet är teknologsektionens högsta beslutande organ.

\subsection{Sammanträden}

\subsubsection{Sammanträden}
Det skall hållas fyra ordinarie sektionsmöten varje år, ett per läsperiod. Utöver detta kan extra sektionsmöten hållas.

\subsection{Utlysande}
\label{sec:utlysande}

\subsubsection{Kallelse}
Sektionsmötet sammanträder på kallelse av sektionsstyrelsen.

\subsubsection{Utlysningsrätt}
Rätt att hos sektionsstyrelsen begära utlysande av sektionsmöte tillkommer:

\begin{itemize}
	\item Medlem i styrelsen
	\item Kårens inspektor
	\item Kårstyrelsen
	\item Sektionens inspektor
	\item Sektionens revisorer
	\item Sektionsmedlem, förutsatt att minst 25 sektionsmedlemmar stödjer förslaget
\end{itemize}

Efter utlysning skall sektionsmötet hållas inom tolv läsdagar

\subsubsection{Utlysning}
Sektionsmötet skall utlysas minst tio dagar i förväg genom att kallelse enligt reglemente anslås. Inkomna motioner och propositioner skall anslås minst tre läsdagar i förväg. Övriga möteshandlingar skall anslås enligt stadga eller reglemente.

\subsection{Åligganden}

\subsubsection{Första ordinarie höstmötet}
Det åligger sektionsmötet att på första ordinarie höstmötet välja valberedning, revisorer samt behandla års- och revisionsberättelse och ansvarsfrihet för föregående läsårs sektionsstyrelse och studienämnd. Det åligger sektionsmötet att fastställa budget samt verksamhetsplan för sektionen.

\subsubsection{Andra ordinarie vårmötet}
Det åligger sektionsmötet att på andra ordinarie vårmötet välja medlemmar till sektionsstyrelsen och studienämnd. Det åligger sektionsmötet att fastställa preliminär budget samt verksamhetsplan för sektionen. Det åligger sektionsmötet att på andra ordinarie vårmöte sektionsavgift för nästkommande läsår.

\subsubsection{Sektionskomittéer}
Det åligger sektionsmötet att välja medlemmar till samtliga sektionskommittéer på det sektionsmöte som reglementet anger. Det åligger sektionsmötet att behandla ansvarsfrihet för samtliga sektionskommittéer.

\subsection{Beslutförhet}

\subsubsection{Beslutsmässighet}
Sektionsmötet är beslutsmässigt om mötet är behörigt utlyst enligt \ref{sec:utlysande}

\subsubsection{Undantag}
Om färre än 30 medlemmar är närvarande då beslut skall fattas, kan detta endast ske om ingen yrkar på bordläggning. Detsamma gäller beslut i frågor som ej varit anslagna tre läsdagar i förväg.

\subsection{Motion}

\subsubsection{Motion}
Medlem som önskar ta upp motion på sektionsmötet skall tillhandahålla motionen till sektionsstyrelsen senast fem läsdagar före sektionsmötet.

\subsection{Överklagande}

\subsubsection{Överklagande}
Beslut av sektionsmötet som strider mot kårens eller sektions stadga, reglemente eller policy får undanröjas av kårstyrelsen. Sådant beslut kan tas upp till prövning om det begärs av en kårmedlem då det rör kårens stadga eller en sektionsmedlem då det rör sektionens stadga.

\subsection{Omröstning}

\subsubsection{Fullmakt}
Röstning med fullmakt får ej ske.

\subsubsection{Omröstningsförfarande}
Omröstning skall ske öppet, om ej sluten votering begärs. Vid lika röstetal har mötesordföranden utslagsröst utom vid personval då lotten avgör.

\subsubsection{Beslut}
Om inget annat anges antas samtliga beslut med enkel majoritet.

\subsection{Närvaro-, yttrande-, förslags- och rösträtt}

\subsubsection{Närvaro- och yttranderätt tillkommer:}
\begin{itemize}
	\item Sektionsmedlem
	\item Hedersmedlem
	\item Kårstyrelseledamöter
	\item Kårens inspektor
	\item Sektionens revisorer
	\item Av mötet adjungerade icke medlemmar
\end{itemize}

\subsubsection{Förslags- och rösträtt}
Förslags- och rösträtt tillkommer endast sektionsmedlemmar.

\subsection{Protokoll}

\subsubsection{Protokoll}
Sektionsmötesprotokoll skall justeras av två av mötet valda justeringspersoner. Justerat protokoll skall korrekt anslås, enligt \ref{sec:protokoll:anslagning}, senast tio läsdagar efter mötet.