\documentclass[11pt, includeaddress]{classes/cthit}
\usepackage{titlesec}

\titleformat{\paragraph}[hang]{\normalfont\normalsize\bfseries}{\theparagraph}{1em}{}
\titlespacing*{\paragraph}{0pt}{3.25ex plus 1ex minus 0.2ex}{0.7em}

\graphicspath{ {images/} }

\begin{document}

\title{Uppförandepolicy}
\approved{2009--05--09}
\editorial{2015--09--04}
\maketitle

\thispagestyle{empty}

\newpage

\makeheadfoot%

%Rubriksnivådjup
\setcounter{tocdepth}{2}
%Sidnumreringsstart
\setcounter{page}{1}
\tableofcontents

\newpage

\section{Syfte}
Denna policy syftar till att definiera de riktlinjer som sektionens medlemmar och aktiva i föreningens kommittéer bör ta hänsyn till i sina kontakter med övriga medlemmar och organ i kåren. Vidare definieras den straffskala som sektionen tillämpar vid överträdelser. 

\section{Bemötande mot andra sektionsmedlemmar, föreningar, kommittéer och sektioner}
Som medlem av sektionen och/eller föreningsmedlem skall man:

\begin{itemize}
	\item Uppträda respektfullt gentemot sektionens föreningar samt kommittéer, deras arbete och arrangemang.
	\item Uppträda respektfullt gentemot andra sektionsmedlemmar.
 	\item Uppträda respektfullt gentemot andra sektioner, deras föreningar samt kommittéer, deras arbete och arrangemang.

\end{itemize}

\section{Skötsel av lokaler}
Som medlem i sektionen eller föreningsmedlem under sektionen:

\begin{itemize}
	\item Skall en behandla sektionens lokaler samt närliggande lokaler med varsamhet.
	\item Skall en under inga omständigheter sova i någon av sektionen tillhörande lokaler. Envar skall även i så stor mån som möjligt se till att inga andra studenter sover i någon av sektionens lokaler.
 	\item Har envar, om en har tillträde till föreningsutrymmen, ansvar att se till att dessa sköts på ett acceptabelt sätt och att de hålls i ett gott skick.
	
\end{itemize}

Som förenings- eller kommittéaktiv i sektionen:

\begin{itemize}
	\item Har en ansvar att föregå med gott exempel i skötseln av sektionens lokaler samt närliggande lokaler.
	\item Skall en vid inträffande av olyckshändelse, skadegörelse, nedskräpning eller liknande händelse rapportera detta till lämplig instans samt sektionens styrelse.
	
\end{itemize}

\section{Uppförande under sektionsarrangemang}
Som sektionsarrangemang räknas arrangemang dit hela eller stora delar av sektionen äger tillträde,
såsom pubkvällar, samtliga arrangemang under mottagningen samt gasquer anordnade av sektionens
föreningar.

\begin{itemize}
	\item På sektionsarrangemang skall ej förekomma olämpligt beteende som kan upplevas som kränkande och/eller strider mot någon av sektionens eller studentkårens policies.
	\item Sektionsarrangemang skall icke medföra skada eller onödigt slitage på lokalerna det arrangeras i.
 	\item Som aktiv medlem av sektionen och arrangör ansvarar man för att personer som genom sitt beteende stör eller förhindrar arrangemanget stoppas samt att nödvändiga åtgärder vidtas.
	\item Incidenter som sker under arrangemang i sektionens regi skall rapporteras till nöjeslivsansvarig för sektionen, samt annan lämplig instans.

\end{itemize}

\section{Sektionens straffskala}
Vid eventuell överträdelse av sektionens policies kring uppförande samt beteende (Uppförandepolicy
för medlemmar och föreningsaktiva inom IT-sektionen) skall följande straffskala tas i bruk.

\subsection{Medlem av sektionen}
\begin{itemize}
	\item Varning
	\item Avstängning Hubben (1 mån)
 	\item Avstängning Hubben (permanent)
\end{itemize}

\subsection{Förenings/kommittéaktiv under sektionen}
\begin{itemize}
	\item Varning
	\item Representationsförbud (1 mån)
 	\item Representationsförbud (3 mån)
	\item Utesluten ur förening/kommitté
\end{itemize}

\subsection{Anmärkningar}
\begin{itemize}
	\item Anmärkning 1. Notera att en så kallad varning ej slutar att gälla efter en viss tid.
	\item Anmärkning 2. För en sektionsmedlem i en förening eller kommitté gäller båda skalorna samtidigt; exempelvis kan överträdelse från kommittéaktiv bestraf as med både avstängning och representationsförbud.
 	\item Anmärkning 3. “Representationsförbud” innebär både förbud av representation av förening eller kommitté på sektionen samt förbud av representation av sektionen som helhet.
	\item Anmärkning 4. Vid incident avgör styrelsen i samråd med involverade parter hur straffskalan skall tolkas och appliceras.
\end{itemize}

\end{document}
