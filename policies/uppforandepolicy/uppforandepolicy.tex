\documentclass[11pt, includeaddress]{classes/cthit}
\usepackage{titlesec}

\titleformat{\paragraph}[hang]{\normalfont\normalsize\bfseries}{\theparagraph}{1em}{}
\titlespacing*{\paragraph}{0pt}{3.25ex plus 1ex minus 0.2ex}{0.7em}

\graphicspath{ {images/} }

\begin{document}

\title{Uppförandepolicy}
\approved{2009--05--09}
%\editorial{2015--09--04}
\editorial{2020--05--14}
\maketitle

\thispagestyle{empty}

\newpage

\makeheadfoot%

%Rubriksnivådjup
\setcounter{tocdepth}{2}
%Sidnumreringsstart
\setcounter{page}{1}
\tableofcontents

\newpage

\section{Syfte}
Denna policy syftar till att definiera de riktlinjer som sektionens medlemmar och aktiva i sektionens organ bör ta hänsyn till i sina kontakter med övriga medlemmar och organ i kåren. Vidare definieras de påföljder som sektionen tillämpar vid överträdelser. 

\section{Bemötande mot andra sektionsmedlemmar, föreningar, kommittéer och sektioner}
Som medlem av sektionen och/eller föreningsmedlem skall man:

\begin{itemize}
	\item Uppträda respektfullt gentemot sektionens föreningar samt kommittéer, deras arbete och arrangemang.
	\item Uppträda respektfullt gentemot andra sektionsmedlemmar samt sektionens alumner.
 	\item Uppträda respektfullt gentemot andra sektioner, deras föreningar samt kommittéer, deras arbete och arrangemang.

\end{itemize}

\section{Skötsel av lokaler}
Som medlem i sektionen eller föreningsmedlem under sektionen:

\begin{itemize}
	\item Skall en behandla sektionens lokaler samt närliggande lokaler med varsamhet.
	\item Skall en under inga omständigheter sova i någon av sektionen tillhörande lokaler.
 	\item Skall alla med tillgång till sektionens föreningsutrymmen ansvara för att se till att se till att dessa sköts på ett acceptabelt sätt och att de hålls i ett gott skick.
	
\end{itemize}

Som förenings- eller kommittéaktiv i sektionen:

\begin{itemize}
	\item Har en ansvar att föregå med gott exempel i skötseln av sektionens lokaler samt närliggande lokaler.
	\item Skall en vid inträffande av olyckshändelse, skadegörelse, medveten nedskräpning eller liknande händelse rapportera detta till lämplig instans samt sektionens styrelse.
	
\end{itemize}

\section{Uppförande under sektionsarrangemang}
Som sektionsarrangemang räknas samtliga arrangemang som arrangeras på sektionen.

\begin{itemize}
	\item På sektionsarrangemang skall ej förekomma olämpligt beteende som kan upplevas som kränkande och/eller strider mot någon av sektionens eller studentkårens policies.
	\item Sektionsarrangemang skall icke medföra skada eller onödigt slitage på lokalerna det arrangeras i.
 	\item Som arrangör av sektionsarrangemang ansvarar man för att personer som genom sitt beteende stör eller förhindrar arrangemanget stoppas samt att nödvändiga åtgärder vidtas.
	\item Incidenter som sker under sektionsarrangemang skall rapporteras till sektionens styrelse, samt annan lämplig instans.

\end{itemize}

\section{Sektionens påföljder}
Vid överträdelse av någon av sektionens policys eller bestämmelser gäller följande.

\begin{itemize}
    \item Innan beslut tas om hurvida en påföljd skall utdelas skall styrelsen föra en diskussion med involverade parter. 
    \item Berörda personer har alltid rätt att överklaga beslut om påföljder till sektionsstyrelsen samt sektionsmötet. Hubbenförbud kan dessutom överklagas till programledningen. 
    \item Vid samtliga applicerbara påföljder avgör sektionsstyrelsen till vilken grad samt hur länge påföljden skall gälla, dessa ska vara tydliga vid fastställandet av påföljden.
    \item Inga påföljder får gälla längre än ett år.
\end{itemize}

\subsection{Förteckning påföljder}
\begin{itemize}
    \item \textbf{Varning} -- Med varning menas att ingen ytterliggare åtgärd kommer vidtas men att vid framtida överträdelser kan strängare påföljder delas ut för mindre överträdelser.
    
    \item \textbf{Arrangemangsförbud} -- Med arrangemangsförbud menas att personen i fråga ej får delta på sektionsarrangemang med undantag av sektionsmöten.
    
    \item \textbf{Hubbenförbud} -- Med hubbenförbud menas att personen i fråga ej får vistas i hubben samt att deras tillträde till hubben kan dras in. Hubbenförbud gäller även under arrangemang i Hubben.
    
    \item \textbf{Representationsförbud} -- Med representationsförbud menas förbud mot representation av förening eller kommitté på sektionen samt förbud mot representation av sektionen som helhet.
    
    \item \textbf{Mottagningsförbud} -- Med mottagningsförbud menas att personen i fråga ej får agera phadder på sektionen samt har arrangemangsförbud och representationsförbud på alla mottagningsarrangemang.
    
    \item \textbf{Avstängning från kommitté} -- Med avstängning från kommitté menas att en person blir avstängd från en eller flera kommittéer som denna är aktiv inom.
    
    \end{itemize}
    
\subsection{Sektionsorgans befogenheter}
Sektionens organ äger rätten att i samråd med styrIT svartlista personer från sina arrangemang. Svartlistning får gälla som längst tills dess att organet i frågas verksamhetsår tar slut.

\end{document}
