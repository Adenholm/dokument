\documentclass[11pt, includeaddress]{classes/cthit}
\usepackage{titlesec}
\usepackage{verbatimbox}

\titleformat{\paragraph}[hang]{\normalfont\normalsize\bfseries}{\theparagraph}{1em}{}
\titlespacing*{\paragraph}{0pt}{3.25ex plus 1ex minus 0.2ex}{0.7em}

\graphicspath{ {images/} }

\begin{document}

\title{Miljöpolicy}
\approved{2013--05--16}
\maketitle

\thispagestyle{empty}

\newpage

\makeheadfoot%

%Rubriksnivådjup
\setcounter{tocdepth}{2}
%Sidnumreringsstart
\setcounter{page}{1}
\tableofcontents

\newpage

\section{Syfte}
Att aktivt arbeta för ett hållbart samhälle blir av allt större vikt, inte minst inom IT­branschen.
Syftet med denna policy är att bidra till att sektionen anammar ett långsiktigt miljöansvarigt
tänkande inom så många områden som möjligt.


\section{Områden}
\subsection{Inköp}
I de fall av inköp där det är möjligt att välja ekologiska alternativ, bör detta göras om det är
ekonomiskt möjligt.
I de fall det är praktiskt möjligt, bör second hand­varor väljas över nyproducerade.

\subsubsection{Kaffe}
Allt kaffe som köps in av sektionen ska, om möjligt, vara ekologiskt och rättvisemärkt.

\subsection{Mat}
All mat under arrangemang arrangerade av eller finansierade av sektionen eller dess föreningar skall vara lakto-ovo-vegetarisk. Lakto-ovo-vegetarisk definieras som vegetarisk (icke-animalisk) mat samt mejeriprodukter och ägg.

Arrangören kan välja att det ska finnas alternativ som inte följer ovan punkt, som alltså inte är lakto-ovo-vegetarisk. Gästen måste då aktivt välja detta alternativ. Gör gästen inget val skall maten följa punkten ovan.

Om det finns alternativ som inte är lakto-ovo-vegatariskt, ska det inte vara ekonomiskt fördelaktigt att välja detta alternativ.

\subsection{Resor}
Resor för inköp bör i så stor mån som möjligt samordnas mellan sektionens olika organ.

Vid resor för inköp bör en avvägning göras mellan den ekonomisk vinningen av att handla hos en
grossist med lägre priser och den miljömässiga vinningen av att handla från lokala butiker som
kräver kortare, eller inga, resor.

Resor med lokaltrafik bör väljas över resor med personbil i de fall detta är praktiskt möjligt.

\subsection{Återvinning}
Vid arrangemang i sektionens namn bör källsortering ske i så stor utsträckning som möjligt.

\subsection{Digitala System}
Vid inköp eller uppdateringar av sektionens digitala system bör hänsyn tas till
energiförbrukningen vid användning, såväl som pris och prestanda.

\subsection{Övriga energiförbrukande produkter}
Sektionens energiförbrukning bör vara så låg som är rimligt. Detta innebär exempelvis att;
Elektroniska produkter bör vara avstängda då de inte används.
Lågenergilampor bör väljas framför andra mer energikrävande alternativ.


\end{document}

