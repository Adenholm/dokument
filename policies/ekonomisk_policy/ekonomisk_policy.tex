\documentclass[11pt, includeaddress]{classes/cthit}
\usepackage{titlesec}
\usepackage{verbatimbox}

\titleformat{\paragraph}[hang]{\normalfont\normalsize\bfseries}{\theparagraph}{1em}{}
\titlespacing*{\paragraph}{0pt}{3.25ex plus 1ex minus 0.2ex}{0.7em}

\graphicspath{ {images/} }

\begin{document}

\title{Ekonomisk policy}
\approved{2012--12--06}
\revisioned{2014--02--26}
\maketitle

\thispagestyle{empty}

\newpage

\makeheadfoot%

%Rubriksnivådjup
\setcounter{tocdepth}{2}
%Sidnumreringsstart
\setcounter{page}{1}
\tableofcontents

\newpage

\section{Syfte}
Denna policy syftar till att underlätta och ge kontinuitet i det ekonomiska arbetet på sektionen genom att samla riktlinjer för hur ekonomiska resurser bör användas.

\section{Anmärkningar}
\begin{itemize}
	\item Anmärkning 1: Alla siffror som avser svenska kronor i detta dokument är beräknade inklusive moms á 12,5\% för mat/dryck och 25\% för annat. 
	\item Anmärkning 2: Detta dokument rör sektionens kommittéer samt sektionsstyrelsen \STYRIT{} och studienämnden \SNIT{}.
	\item Anmärkning 3: Om anledning finns att frångå policyn kan godkännande fås av \STYRIT{}.
	\item Anmärkning 4: Ett godkännande från styrelsen lämnas skriftligen från styrITs kassör som skall föregås av styrelsebeslut.
\end{itemize}

\section{Slarv}
Slarv avser alla oförutsedda utgifter i form av böter, kontrollavgifter etc. som tillkommer i samband med kommitténs verksamhet. \\
Slarv betalas av antingen kommitté eller privatperson efter beslut av styrelsen. I detta beslut tas i beaktande huruvida händelsen bedöms ha inträffat på grund av vårdslöshet eller misstag. Har samma kommitté eller person gjort sig skyldig till flera överträdelser tas även detta i beaktning vid framtida bedömningar.

\section{Mat under arrangemang}
Mat under arrangemang avser mat och dryck som förtärs av arbetande i samband med kommitténs verksamhet, där arbetande jobbar minst 4 timmar i sträck. Beroende på arbetspassets längd får varje person bjudas på mat till ett värde av x kronor enligt nedanstående tabell.

\addvbuffer[\the\baselineskip]{\begin{tabular}{ l  c  c  c  c}
	\centering
	Timmar & [0-4] & (4-9] & (9-13] & (13-$\infty$] \\
	\hline
	Kronor & 0 & 30 & 60 & 90 \\
\end{tabular}}

Summorna är baserade på vad som anses vara en rimlig måltidskostnad. 

\section{Aspning}
Då aspar arbetar under arrangemang eller förberedelse av dessa skall de anses likställda med förtroendevalda vad gäller ovanstående punkter. \\

Kostnaden för aspning ska ej överstiga 3000 kronor per kommitté och verksamhetsår. 
Aspning för samtliga kommittéer bekostas av styrelsen.

\section{Överlämning}
Kostnaden för överlämning ska ej överstiga 3000 kronor per kommitté.

\section{Representation}

\subsection{Profilering}
Profilering avser profil- och arbetskläder samt tygmärken och andra inventarier som köps in med syfte att marknadsföra organisationen. \\

Total kostnad för profil- och arbetskläder får ej överstiga 500 kronor per person och verksamhetsår. För overallskommittéerna (\NOLLKIT{}, \PRIT{}, \SEXIT{}) får ytterligare en kostnad på 500 kronor per medlem och verksamhetsår tillkomma. För samtliga organ inom sektionen med ett behov av en större budget för profileringsändamål får detta tillkomma med styrelsens godkännande.

\subsection{Intern representation}
Intern representation avser tillfälliga och kortvariga aktiviteter som sker inom kommittén och är direkt kopplade till verksamheten. Exempel på detta är teambuilding och liknande. \\

Vid intern representation gäller följande kostnadstak: Mat och dryck får inte överstiga 90 kr/person och tillfälle. Ingen andel av denna summa får läggas på alkoholhaltiga drycker. \\

Om det kan motiveras så finns det möjlighet att lägga upp till 180 kr per person och tillfälle på kringkostnader, så som lokalhyror, resekostnader, biljettkostnader vid ar-
rangemang etc. Utgifter av denna typ skall i förhand godkännas av styrelsen. \\

Den totala kostnaden för intern representation får inte överstiga 630 kr/person och verksamhetsår. 

\subsection{Extern representation}
Extern representation avser representation med parter utanför kommittén. Extern representation bör ej förekomma om det inte anses nödvändigt för att ett samarbete med annan part skall fortsätta. Ett exempel på extern representation är tackkalas för puffar eller phaddrar. Utgifter som dessa måste godkännas av styrelsen.


\section{Inventarielista}
Inventarier syftar på större inköp eller objekt som har ett speciellt kontinuerligt värde för verksamheten. Ett exempel i dagsläget kan vara \NOLLKIT{}s bollhav som oavsett finansiellt värde har ett stort sentimentalt värde för kommittén.

Samtliga kommittéer skall föra inventarielista över dess ägodelar. På denna lista skall finnas:

\begin{itemize}
	\item Beskrivning av artikeln
	\item Inköpsdatum
	\item Inköpspris
	\item Kopia på inköpskvitto
\end{itemize}


\section{Utomstående personal}
Utomstående personal kan t ex avse musiker. Sådan typ av personal får endast anlitas om denne innehar F-skattesedel.

\section{Äskningar}
En äskning är en förfrågan till styrelsen om finansiella medel för att utföra en aktivitet eller göra ett inköp till nytta för sektionen och dess medlemmar. Görs äskningen av en kommitté så skall utgiften vara av en sådan art som inte täcks in av kommitténs huvudsakliga verksamhet eller som utan finansiell hjälp ej är genomförbar. \\

Äskningar skall skickas via mail till styrelsen i enlighet med mall som finns tillgänglig på sektionens hemsida. Varje äskning bedöms individuellt av styrelsen enligt ovanstående ramar inom 10 läsdagar.

\section{Större inköp}
Större inköp avser köp av enskilt objekt som uppgår till ett belopp som överstiger 1500 kr. Dessa inköp skall godkännas av styrelsen.

\section{Förvaring av kontanter i sektionslokalen}
Den sammanlagda summan i sektionslokalen bör ej överstiga 50 000 kronor. \\
Dagskassan före och efter ett arrangemang för en förening eller kommitté får överstiga den sammanlagda summan i sektionslokalen men får inte förvaras i lokalen i mer än ett av nedanstående tabell angivet antal dagar.

\begin{table}[h]
\centering
\begin{tabular}{  l  c }
	{\large{\textbf{Summa pengar}}} & {\large{\textbf{Antal arbetsdagar}}} \\
	\hline
	{10 000 - 30 000kr} & {5} \\
	{30 001 - 50 000kr} & {3} \\
	{> 50 000kr} & {1} \\
\end{tabular}
\end{table}

\section{Bokföring}
Kassörer ska uppvisa bokföring för sektionens lekmannarevisorer inför alla ordinarie sektionsmöten. Om kassör, utan god anledning, inte lämnar in sin bokföring (så väl digital som fysisk) i tid eller om den bokföring som lämnats in ej enligt lekmannarevisorerna håller tillräckligt hög kvalitet får berörd kommitté spenderings- och arrangemangsstopp fram tills dess att bokföring återigen uppvisats för och godkänts av lekmannarevisorerna.

\end{document}
