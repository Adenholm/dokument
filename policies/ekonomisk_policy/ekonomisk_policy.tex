\documentclass[11pt, includeaddress]{../../classes/cthit}
\usepackage{titlesec}
\usepackage[swedish]{babel}
\usepackage{verbatimbox}

\titleformat{\paragraph}[hang]{\normalfont\normalsize\bfseries}{\theparagraph}{1em}{}
\titlespacing*{\paragraph}{0pt}{3.25ex plus 1ex minus 0.2ex}{0.7em}

\graphicspath{ {../../images/} }

\begin{document}

\title{Ekonomisk policy}
\approved{2012--12--06}
\maketitle

\thispagestyle{empty}

\newpage

\makeheadfoot%

%Rubriksnivådjup
\setcounter{tocdepth}{2}
%Sidnumreringsstart
\setcounter{page}{1}
\tableofcontents

\newpage

\begin{itemize}
	\item Anmärkning 1: Alla siffror som avser svenska kronor i detta dokument är beräknade inklusive moms à 12,5\% för mat/dryck och 25\% för annat. Siffrorna är i de flesta fall baserade på de skattefria nivåer som finns i Sverige, och skall därför aldrig överskridas.
	\item Anmärkning 2: Detta dokument rör sektionens sektionskommittéer samt styrelsen \STYRIT och studienämnden \SNIT. 
	\item Anmärkning 3: Ett godkännande från styrelsen lämnas skriftligen från \STYRIT{}s kassör som skall föregås av styrelsebeslut
\end{itemize}

Vid oklarheter kring denna text, kontakta styrelsen på 
\MYhref{mailto:styrit@chalmers.it}{styrit@chalmers.it}


\section{Ekonomiskt ansvar}
Ordförande och kassör har gemensamt ansvar för kommitténs ekonomi.


\section{Slarv}
Slarv avser alla oförutsedda utgifter i form av böter, kontrollavgifter etc. som tillkommer i samband med kommitténs verksamhet. Slarv betalas av antingen kommitté eller privatperson efter beslut av styrelsen. I detta beslut tas i beaktande huruvida händelsen bedöms ha inträffat på grund av vårdslöshet eller misstag. Har samma kommitté eller person gjort sig skyldig till flera överträdelser tas även detta i beaktning vid framtida bedömningar.


\section{Mat under arrangemang}
Mat under arrangemang avser mat och dryck som förtärs av arbetande i samband med kommitténs verksamhet, där arbetande jobbar minst 5 timmar i sträck. Beroende på arbetspassets längd får varje person bjudas på mat till ett värde av x kronor enligt nedanstående tabell.

\addvbuffer[\the\baselineskip]{\begin{tabular}{ l  c  c  c  c}
	\centering
	Timmar & [0-5) & (5-9] & (9-13] & (13-24] \\
	\hline
	Kronor & 0 & 30 & 60 & 90 \\
\end{tabular}}

Summorna är baserade på vad som anses vara en rimlig måltidskostnad. Utöver detta får den sammanlagda totala kostnaden ej överstiga 700 kronor för varje enskilt arrangemang. Därutöver får denna kostnad ej överskrida 500 kronor per person och år. Om ovanstående gränser bedöms för låga för ett specifikt arrangemang så finns det möjlighet att göra en specialbedömning tillsammans med styrelsen. Om inte denna kostnad bedöms möjlig att finansiera med kommitténs verksamhet finns det möjlighet att äska om ytterligare medel från styrelsen. 


\section{Aspning}
Då aspar arbetar under arrangemang eller förberedelse av dessa skall de anses likställda med sittande medlemmar vad gäller ovanstående punkter. Kostnaden för aspning bör ej överstiga 3000 kronor per kommitté och verksamhetsår. Om kommitté har för avsikt att överstiga detta belopp skall detta godkännas av styrelsen.


\section{Representation}

\subsection{Profilering}
Profilering avser profil- och arbetskläder samt tygmärken och andra inventarier som köps in med syfte att marknadsföra organisationen. Total kostnad för profil- och arbetskläder får ej överstiga 500 kronor per person och verksamhetsår. För overallskommittéerna (\NOLLKIT, \PRIT, \SEXIT) får även en kostnad på 500 kronor per kommittémedlem tillkomma. Övriga utgifter i profileringssyfte skall godkännas av styrelsen. 

\subsection{Intern representation}
Intern representation avser tillfälliga och kortvariga aktiviteter som sker inom kommittén och är direkt kopplade till verksamheten. Exempel på detta är teambuilding och liknande. Vid intern representation gäller följande kostnadstak: Mat och dryck får inte överstiga 90 kr/person och tillfälle. Ingen andel av denna summa får läggas på alkoholhaltiga drycker. Om det kan motiveras så finns det möjlighet att lägga upp till 180 kr per person och tillfälle på kringkostnader, så som lokalhyror, resekostnader, biljettkostnader vid arrangemang etc. Utgifter av denna typ skall i förhand godkännas av styrelsen.

\subsection{Extern representation}
Extern representation avser representation med parter utanför kommittén. Extern representation bör ej förekomma om det inte anses nödvändigt för att ett samarbete med annan part skall fortsätta. Ett exempel på godkänd extern representation är tackkalas för puffar eller phaddrar. Även detta måste godkännas av styrelsen.


\section{Inventarielista}
Inventarier syftar på större inköp eller objekt som har ett speciellt kontinuerligt värde för verksamheten. Ett exempel i dagsläget kan vara \NOLLKIT{}s bollhav som oavsett finansiellt värde har ett stort sentimentalt värde för kommittén. Samtliga kommittéer skall föra inventarielista över dess ägodelar. På denna lista skall finnas beskrivning av artikeln, inköpsdatum och inköpspris. Utöver detta skall även kopia på inköpskvitto finnas.


\section{Utomstående personal}
Utomstående personal kan t ex avse musiker. Sådan typ av personal får endast anlitas om denne innehar F-skattesedel. 


\section{Äskningar}
En äskning är en förfrågan till styrelsen om finansiella medel för att utföra en aktivitet eller göra ett inköp till nytta för sektionen och dess medlemmar. Görs äskningen av en kommitté så skall utgiften vara av en sådan art som inte täcks in av kommitténs huvudsakliga verksamhet eller som utan finansiell hjälp ej är genomförbar. 

Äskningar skall skickas via mail till styrelsen i enlighet med mall som finns tillgänglig på styrelsens hemsida. Varje äskning bedöms individuellt av styrelsen enligt ovanstående ramar. 


\section{Större inköp}
Större inköp avser köp av enskilt objekt som uppgår till ett belopp som överstiger 1500 kr. Dessa inköp skall godkännas av styrelsen.

\end{document}