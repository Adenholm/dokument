\documentclass[11pt, includeaddress]{../../classes/cthit}
\usepackage{titlesec}
\usepackage[swedish]{babel}

\titleformat{\paragraph}[hang]{\normalfont\normalsize\bfseries}{\theparagraph}{1em}{}
\titlespacing*{\paragraph}{0pt}{3.25ex plus 1ex minus 0.2ex}{0.7em}

\graphicspath{ {../../images/} }

\begin{document}

\title{Lokalpolicy}
\approved{2013--02--28}
\maketitle

\thispagestyle{empty}

\newpage

\makeheadfoot%

%Rubriksnivådjup
\setcounter{tocdepth}{2}
%Sidnumreringsstart
\setcounter{page}{1}
\tableofcontents

\newpage

\emph{Förutom IT­-sektionens egen lokalpolicy, lyder också sektionens medlemmar under kårens centrala lokalpolicy samt dispositionsavtalet mot Akademiska Hus. Se nedan:}

%MYhref definerad i classen, ger möjlighet att byta färg mm
\MYhref{http://www.chs.chalmers.se/sites/default/files/uploads/Lokalpolicy.pdf}{Kårens lokalpolicy}\\
\MYhref{http://styrit.chalmers.it/documents//regulation/Lokalpolicy.pdf}{Dispositionsavtal}


\section{Förteckning}
Sektionens lokaler består av:
\begin{itemize}  
  \item \HUBBEN 
  	\begin{itemize}  
	  \item Stora rummet
	  \item Köket
	  \item Grupprummet
	  \item Studierummet
	  \item Föreningsrummet
	  \item Chillen
	\end{itemize}
\end{itemize}


\section{Syfte}
Sektionens lokaler är till för medlemmar i Teknologsektionen Informationsteknik.
\HUBBEN är i första hand avsedd för studier och som lunchlokal. Det bör hållas en rimlig ljudnivå under studietid (8.00 - 17.00 på studiedagar). Under den tiden får heller ingen alkohol förtäras i lokalen. Vidare är \HUBBEN avsedd för arrangemang för sektionens medlemmar och skall vara ett ställe där alla medlemmar kan umgås. Utöver detta dokument regleras lokalens nyttjande av dispositionsavtalet.

\subsection{Grupprummet}
Grupprummet är främst avsett för individuella studier eller grupparbeten. Under luncher och efter klockan 17:00 kan grupprummet också fungera som möteslokal för sektionens föreningar och kommittéer. Ljudnivån ska vara rimlig under studietid.

\subsection{Studierummet}
Studierrummet är främst avsett för studier och ska alltid vara tillgängligt för alla medlemmar av sektionen (undantag kan göras vid större arrangemang eller vid godkännande av \STYRIT). Här ska det hållas en låg ljudnivå. Fest i studierummet får inte förekomma. Ej heller under större arrangemang då studierummet enligt dispositionsavtalet ej tillåter fest.

\subsection{Föreningsrummet}
Föreningsrummet är endast avsett som förvaringsutrymme för sektionens kommittéer och föreningar.

\subsection{Chillen}
Chillen ingår för stunden inte i  det vistelsebara \HUBBEN.


\section{Tillträde}
\begin{itemize}
	\item Alla IT-teknologer har tillträde till \HUBBEN såvida ej serveringstillstånd gäller. Om \HUBBEN är bokad bör detta dock respekteras.
	\item I undantagsfall har \STYRIT rätt att frånta enskild teknolog tillträde till \HUBBEN.
	\item Det är inte under några omständigheter tillåtet att sova i \HUBBEN.
\end{itemize}


\section{Hyra}
\subsection{Lokaler}
\HUBBEN får ej hyras ut.

\subsection{Inventarier}
Hubbens inventarier får hyras ut i samband med att \HUBBEN lånas ut till extern part.


\section{Boka}
\HUBBEN får bokas av IT-­sektionens förtroendevalda efter kl 17.00 på studiedagar och dygnet runt på helger. Dessa bokningar får endast göras för sektionsnyttiga ändamål. Undantag gäller för tentaveckor (även omtentaveckor), då \HUBBEN inte får bokas annat än till studierelaterade arrangemang, samt under mottagningen då \HUBBEN är bokningsbar även under dagtid. Tentaveckan räknas från lördag kl 08.00 till nästföljande lördag kl 13.00. Grupprummet bokas separat från övriga \HUBBEN. Styrelsen samt lokalansvarig kommitté har rätt att häva bokningar om de ej anses lämpliga. 

\subsection{Grupprummet}
Grupprummet får bokas under lunchtid och efter kl 17.00. Undantag gäller för tentaveckor, då grupprummet inte för bokas annat än till pluggrelaterade arrangemang.

\subsection{Studierummet}
Studierummet får ej bokas.

\subsection{Chillen}
Chillen ingår för stunden inte i det bokningsbara \HUBBEN. 


\section{Inventarier}
Hubbens inventarier får gratis nyttjas av medlemmar i IT­-sektionen då dessa har bokat \HUBBEN. Invenatier som införskaffas till \HUBBEN måste vara flamsäkra. Hubbens inventarier får ej lämna lokalen utan medgivande från \PRIT 

\section{Ansvar vid arrangemang}
\begin{itemize}
	\item Arrangör är skyldig att vara väl införstådd med detta dokument, lokalens dispositionsavtal samt studentkårens lokalpolicy. 
	\item Arrangör är ansvarig för skador som uppstår under arrangemang. 
	\item Arrangör är ansvarig för att återställa lokalen till ett acceptabelt skick efter arrangemang. Denna nivå sätts i samspråk med \PRIT vid utlämning.
\end{itemize}


\end{document}