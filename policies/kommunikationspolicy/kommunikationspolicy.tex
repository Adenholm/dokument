\documentclass[11pt, includeaddress]{classes/cthit}
\usepackage{titlesec}

\titleformat{\paragraph}[hang]{\normalfont\normalsize\bfseries}{\theparagraph}{1em}{}
\titlespacing*{\paragraph}{0pt}{3.25ex plus 1ex minus 0.2ex}{0.7em}

\graphicspath{ {images/} }

\begin{document}

\title{Kommunikationspolicy}
\approved{2014--12--11}
\maketitle

\thispagestyle{empty}

\newpage

\makeheadfoot%

%Rubriksnivådjup
\setcounter{tocdepth}{2}
%Sidnumreringsstart
\setcounter{page}{1}
\tableofcontents

\newpage

\section{Syfte}
Syftet med policyn är att skapa en struktur kring hur sektionens organ kommunicerar med sektionens medlemmar, för att långsiktigt öka jämlikheten på sektionen.\\
 \\
Alla medlemmar som är aktiva inom sektionens verksamhet ska vara införstådda med kårens dokument Riktlinjer för Chalmers Studentkårs kommunikation på engelska. Som komplement till kårens riktlinjer gäller även de riktlinjer definierade i denna policy. 

\section{Sektionslokalens anslagstavla}
På sektionslokalens anslagstavla tillåts alla organ på Chalmers affischera. Det är inte tillåtet för företag eller kårföretag att affischera på anslagstavlan utan tillåtelse från arbetsmarknadsgruppen på IT. Affischer får som störst vara av storleken A3 och samma affisch får endast förekomma en gång på anslagstavlan. Det är inte tillåtet att på något sätt modifiera eller ta ned någon annans affisch. Affischers innehåll måste följa de riktlinjer som finns inom kåren och sektionen. \\
 \\
Affischer för evenemang som uppfyller kraven ovan får sitta på anslagstavlan två veckor innan evenemangets datum fram tills dess att evenemangets datum passerat. \\
 \\
Lokalansvarig kommitté samt sektionsstyrelsen innehar befogenhet att ta ned affischer som inte upprätthåller ovanstående krav. 

\section{Sektionsmötet}
Sektionsmötet ska utöver den ordinarie utlysningen även utlysas på engelska med en sammanfattning av dagordningen. Efter sektionsmötet ska en engelsk sammanfattning av de beslut som har fattats finnas tillgänglig för sektionens medlemmar. Denna sammanfattning ska minst innehålla korta beskrivningar om, samt beslut gällande, motioner, propositioner och inval. Båda dessa uppgifter åligger det sektionsstyrelsen att utföra.

\section{Spridning av allmän information}
Information som är riktad till alla sektionens medlemmar samt information som är riktad endast till masterstudenter ska vara på svenska och engelska (eller bara på engelska).

\newpage

\section{PR av evenemang}
De aktiviteter och evenemang som arrangeras av sektionens organ som är till för alla studenter ska marknadsföras på sektionens hemsida. I evenemangets beskrivning, oavsett forum för marknadsföring, bör det vara definierat om språket som kommer att talas under evenemanget är svenska eller engelska.

\subsection*{Fysisk PR}
Med fysisk PR menas den marknadsföring som sker med hjälp av affischer i sektionslokalen eller på campus. Dessa affischer bör vara på engelska.
\subsection*{Digital PR}
Med digital PR menas den marknadsföring som sker digitalt via exempelvis sektionens hemsida eller på sociala nätverk. Denna marknadsföring ska finnas tillgänglig på både svenska och engelska (eller bara på engelska). Detta gäller även om det talade språket på evenemanget är svenska.


\end{document}
